\section{Einleitung}
\subsection{Motivation}
In Anlehnung an die Warenkorbanalyse sollen die vergebenen Themen f�r wissenschaftliche Publikationen analysiert werden,
um so h�ufig miteinander vergebene Themen und Br�ckenthemen zwischen Fachgebieten zu ermitteln.
Zusammen mit der Ermittlung von Koautorengraphen und der Zitationsanalyse ist die Analyse von Themen eine der Hauptmethoden der Bibliometrie.
Eine Untersuchung der Themenvergabe von wissenschaftlichen Publikationen kann interessante Erkenntnisse liefern, um die zeitliche Entwicklung bestimmten Themengebieten zu verfolgen oder verwandte Publikationen zu ermitteln.
\subsection{Problemstellung}
In dieser Arbeit werden die Themen zu wissenschaftlichen Publikationen anhand der ver�ffentlichten Publikationen auf arXiv.org zum 01.10.2011 beispielhaft untersucht.
Das Verfahren, um Beziehungen zwischen Themen zu finden, ist die Assoziationsanalyse.
Durch Assoziationsregeln k�nnen Themen, die oft zusammen auftreten, schnell aufgedeckt werden, bzw. vorhandenen Korrelationen und Abh�ngigkeiten zwischen denen ermittelt werden.
Durch eine geeignete Sortierung nach den verschiedenen Assoziationsregelnkenngr��en - Support, Konfidenz und Lift, k�nnen wir ferner die wichtigsten oder am �ftersten im Datensatz vorkommenden Regeln bestimmen.
Das durch diese Analyse gewonnene Wissen kann sp�ter f�r den Entwurf eines Paperrecommendersysstems oder das Erstellen eines geeigneten und zuverl�ssigen Mappings zwischen verschiedenen wissenschlaftlichen Klassifikationen.
\\
\\
Unser Analyseansatz basiert auf $Paper$ und die Auswertungen wurden mit Hilfe der Programmiersprache $R$ durchgef�hrt.
F�r das Parsen der XML-Daten benutzen wir einen in $Python$ geschriebenen Parser.

\subsection{Aufbau dieses Papers}
Im Weiteren wird dieses Dokument folgenderma�en aufgebaut:\newline
Kapitel 2 stellt die theoretischen Grundlagen der Assoziationsanalyse vor und erkl�rt den apriori-Algorithmus, den wir f�r die Extraktion der Assoziationsregeln in unserer Arbeit benutzen.
Im Kapitel 3 beschreiben wir die Merkmale und Besonderheiten des arxiv.org Datensatzes, sowie wie wir diesen weiter verarbeitet haben.
In Kapitel 4 geben wir eine detaillierte Auswertung der gefundenen Ergebnisse.
Kapitel 5 schl�gt weitere Aspekte vor, die erforscht werden k�nnen, und betrachtet zugleich m�gliche Anwendungen der Forschungsergebnisse.
Kapitel 6 fasst nochmal die Arbeit zusammen.

%-present the subject, it could be with an example
%-define the important words
%-present the hypothesis + arguments etc.
%-describe how the body is organized

%Quite literally, the Introduction must answer the questions, "What was I studying? Why was it an important question? What did we know about it before I did this study? How will this study advance our knowledge?"

