\section{Einleitung}
\subsection{Motivation}
In Anlehnung an die Warenkorbeanalyse sollen die vergebene Themen f�r wissenschaftliche Publikationen analysiert werden,
um so h�ufig miteinander vergebene Themen und Br�ckenthemen zwischen Fachgebieten zu ermitteln.
\subsection{Problemstellung}
Die Themen zu wissenschaftlichen Publikationen werden anhand der ver�ffentlichen Publikationen auf arXiv.org beispielhaft untersucht. Das Verfahren, um Beziehungen zwischen Themen zu finden wird die Assoziationsanalyse sein. Durch Assoziationsregeln k�nnen 
\section{Assozationsanalyse}\cite{Agrawal:1993:MAR:170035.170072}% TODO test um unser Startpaper zu zitieren
Die Assoziationanalyse beschreibt die Suche nach Assoziationsregeln innnerhalb einer Menge von Transaktionen.
Transaktionen sind endliche Teilmengen aus einer sogenannten Itemmenge. Item sind Objekte, die in Transaktionen auftreten
k�nnen. Ein Datensatz zur Assoziationsanalyse besteht, also aus einer Menge von Transaktionen.
\subsection{Assozationsregeln}
Assoziationsregeln sind Implikationen der Form $A \rightarrow B$ und beschreiben den Sachverhalt, dass wenn $A$ auftritt auch $B$ auftritt.
\subsection{Kenngr��en von Assoziationsregeln}
\subsubsection{Support}
Der Support bezeichnet die relative H�ufigkeit des Auftretens der Itemmenge $X$ in einem Datensatz..
\begin{mdef}[Support]
\[
\mathrm{support}(X) = \frac{\text{Anzahl der vorkommen von X im Datensatz}}{\text{Anzahl aller Transaktionen}}
\]
\end{mdef}
\subsubsection{Konfidenz}
Die ist relative H�ufigkeit der Transaktionen in denen X vorkommt zusammen mit Y.
\begin{mdef}[Konfidenz]
\[
\mathrm{conf}(X\rightarrow Y) = \frac{\mathrm{support}(X \cup Y)}{\mathrm{support}(X)} 
\]
\end{mdef}
\subsubsection{Lift}
Der Lift wird f�r die St�rke einer Regeli bzw. Bedeutung herrangezogen. 
\begin{mdef}[Lift]
\[
\mathrm{lift}(X\rightarrow Y) = \frac{ \mathrm{support}(X \cup Y)}{ \mathrm{support}(Y) \cdot \mathrm{support}(X) } 
\]
\end{mdef}
\subsection{apriori-Algorithmus}
Der apriori-Algorithmus wurden 1994 von \cite{Agrawal:1994:FAM:645920.672836}  ver�ffentlicht, der sich fundamental zu denbis dato existierenden Algorithmen zum Finden von Assoziationsregeln unterscheidet. \\
$I = \{i_1, i_2, \ldots i_m\}$ ist eine endliche Menge von Items, die in einer Transaktion vorkommen k�nnen. 
$D = \{t_1, t_2, \ldots , t_n\}$ ist eine Menge von Transaktionen, wobei jede Transaktion $T \subset I$ ein Teilmenge der Menge von Items ist. Weiterhin wird ein minimaler Support $\mathrm{minsupp}$ und eine minimale Konfidenz $\mathrm{minconf}$ definiert, die als Kriterium f�r die Assoziationsregeln verwendet wird. Der Algorithmus wird in zwei Probleme aufgeteilt.
\begin{description}
	\item[Problem 1] Finden aller Teilmengen von Items in die Transaktionen, die mindestens einen Support von $\mathrm{minsupp}$ haben.
	\item[Problem 2] Aus den gefundenen Teilmengen in Problem 1 werden Assoziationsregeln erzeugt. Nur Regeln die mindestens eine Konfidenz von $\mathrm{minconf}$ haben, werden ber�cksichtigt.
\end{description}
\subsubsection{Finden von h�ufigen Mengen}
Sei $L_k$ die Menge aller Mengen mit k Items, die mindestens eine Support von $\mathrm{minsupp}$ und $C_k$ die Menge m�glicher Kandidaten f�r $L_k$. Im ersten Schritt wird die H�ufigkeit aller Items in den Transaktionen aus $D$ berechnet und die Items, die mindestens $\mathrm{minsupp}$ haben bilden $L_1$. In der k-ten Iteration werden neue Kandidaten f�r $L-k$ erzeugt und wenn diese wieder $\mathrm{minsupp}$ �bersteigen in $L_k$ aufgenommen. Der Algorithmus terminiert, wenn keiner der erzeugten Kandidatenmengen in $L_k$ aufgenommen wurde, also $L_k = \emptyset$.
\begin{algorithm}
	\caption{aprior Algorithmus}
	\label{apriori}
\begin{algorithmic}
	\STATE $L_1\gets \{\text{alle Items mit } \mathrm{minsupp}\}$
	\FOR{$k = 2$,$L_{k-1} = \emptyset$, $k++$} 
		\STATE $L_k \gets \mathrm{apriori-gen}(L_{k-1})$
		\FORALL{$t \in D$} 
			\STATE $i \gets i + 1$
		\ENDFOR
	\ENDFOR
\end{algorithmic}
\end{algorithm}
\subsubsection{Erzeugen von Assoziationsregeln}

