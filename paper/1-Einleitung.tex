\section{Einleitung}
\subsection{Motivation}
In Anlehnung an die Warenkorbeanalyse sollen die vergebene Themen f�r wissenschaftliche Publikationen analysiert werden,
um so h�ufig miteinander vergebene Themen und Br�ckenthemen zwischen Fachgebieten zu ermitteln.
\subsection{Problemstellung}
Die Themen zu wissenschaftlichen Publikationen werden anhand der ver�ffentlichen Publikationen auf arXiv.org beispielhaft untersucht. Das Verfahren, um Beziehungen zwischen Themen zu finden wird die Assoziationsanalyse sein. Durch Assoziationsregeln k�nnen 
\section{Assozationsanalyse}\cite{AIS}% TODO test um unser Startpaper zu zitieren
Die Assoziationanalyse beschreibt die Suche nach Assoziationsregeln innnerhalb einer Menge von Transaktionen.
Transaktionen sind endliche Teilmengen aus einer sogenannten Itemmenge. Item sind Objekte, die in Transaktionen auftreten
k�nnen. Ein Datensatz zur Assoziationsanalyse besteht, also aus einer Menge von Transaktionen.
\subsection{Assozationsregeln}
Assoziationsregeln sind Implikationen der Form $A \rightarrow B$ und beschreiben den Sachverhalt, dass wenn $A$ auftritt auch $B$ auftritt.
\subsection{Kenngr��en von Assoziationsregeln}
\subsubsection{Support}
Der Support bezeichnet die relative H�ufigkeit des Auftretens der Itemmenge $X$ in einem Datensatz..
\begin{mdef}[Support]
\[
\mathrm{support}(X) = \frac{\text{Anzahl der vorkommen von X im Datensatz}}{\text{Anzahl aller Transaktionen}}
\]
\end{mdef}
\subsubsection{Konfidenz}
Die ist relative H�ufigkeit der Transaktionen in denen X vorkommt zusammen mit Y.
\begin{mdef}[Konfidenz]
\[
\mathrm{conf}(X\rightarrow Y) = \frac{\mathrm{support}(X \cup Y)}{\mathrm{support}(X)} 
\]
\end{mdef}
\subsubsection{Lift}
Der Lift wird f�r die St�rke einer Regeli bzw. Bedeutung herrangezogen. 
\begin{mdef}[Lift]
\[
\mathrm{lift}(X\rightarrow Y) = \frac{ \mathrm{support}(X \cup Y)}{ \mathrm{support}(Y) \cdot \mathrm{support}(X) } 
\]
\end{mdef}
\subsection{apriori-Algorithmus}
