\section{Assozationsanalyse}
\label{sec:assoziationsanalyse}

\cite{Agrawal:1993:MAR:170035.170072}% TODO test um unser Startpaper zu zitieren
Die Assoziationsanalyse beschreibt die Suche nach Assoziationsregeln innnerhalb einer Menge von Transaktionen. Transaktionen sind endliche Teilmengen einer sogenannten Itemmenge. Items sind Objekte, die in Transaktionen auftreten k�nnen. Ein Datensatz zur Assoziationsanalyse besteht also aus einer Menge von Transaktionen.
\subsection{Assozationsregeln}
Assoziationsregeln sind Implikationen der Form $A \rightarrow B$ und beschreiben den Sachverhalt, dass wenn $A$ auftritt auch $B$ auftritt.
\subsection{Kenngr��en von Assoziationsregeln}
\subsubsection{Support}
Der Support bezeichnet die relative H�ufigkeit des Auftretens der Itemmenge $X$ in einem Datensatz.
\begin{mdef}[Support]
\[
\mathrm{support}(X) = \frac{\text{Anzahl der vorkommen von X im Datensatz}}{\text{Anzahl aller Transaktionen}}
\]
\end{mdef}
\subsubsection{Konfidenz}
Die Konfidenz ist die relative H�ufigkeit der Transaktionen in denen X zusammen mit Y vorkommt. Oder anders formuliert, beschreibt diese, wie oft Y vorkommt, gegeben dass X vorkommt.
\begin{mdef}[Konfidenz]
\[
\mathrm{conf}(X\rightarrow Y) = \frac{\mathrm{support}(X \cup Y)}{\mathrm{support}(X)}
\]
\end{mdef}
\subsubsection{Lift}
Der Lift wird f�r die St�rke einer Regel bzw. deren Bedeutung herrangezogen.
\begin{mdef}[Lift]
\[
\mathrm{lift}(X\rightarrow Y) = \frac{ \mathrm{support}(X \cup Y)}{ \mathrm{support}(Y) \cdot \mathrm{support}(X) }
\]
\end{mdef}
\subsection{apriori-Algorithmus}
Der apriori-Algorithmus wurde 1994 von \cite{Agrawal:1994:FAM:645920.672836}  ver�ffentlicht. Dieser unterscheidet sich fundamental zu denbis dato existierenden Algorithmen zum Finden von Assoziationsregeln. Er besch�ftigt sich mit der folgenden Fragestellung: \\
$I = \{i_1, i_2, \ldots i_m\}$ ist eine endliche Menge von Items, die in einer Transaktion vorkommen k�nnen.
$D = \{t_1, t_2, \ldots , t_n\}$ ist eine Menge von Transaktionen, wobei jede Transaktion $T \subset I$ ein Teilmenge der Menge von Items ist. Weiterhin werden ein minimaler Support $\mathrm{minsupp}$ und eine minimale Konfidenz $\mathrm{minconf}$ definiert, die als Einschr�nkungskriterium f�r die gesuchten Assoziationsregeln verwendet werden. Der Algorithmus wird in zwei Probleme aufgeteilt.
\begin{description}
	\item[Problem 1] Finden aller Teilmengen von Items in den Transaktionen, die mindestens einen Support von $\mathrm{minsupp}$ haben.
	\item[Problem 2] Aus den gefundenen Teilmengen in Problem 1 werden Assoziationsregeln erzeugt. Nur Regeln, die mindestens eine Konfidenz von $\mathrm{minconf}$ haben, werden ber�cksichtigt.
\end{description}
\subsubsection{Problem 1: Finden von h�ufigen Mengen}
Sei $L_k$ die Menge aller Mengen mit k Items, die mindestens einen Support von $\mathrm{minsupp}$ haben, und $C_k$ die Menge m�glicher Kandidaten f�r $L_k$. Im ersten Schritt wird die H�ufigkeit aller Items in den Transaktionen aus dem gesamten Datensatz $D$ berechnet und die Items, die mindestens $\mathrm{minsupp}$ haben, bilden $L_1$. In der k-ten Iteration werden neue Kandidaten f�r $L_k$ erzeugt und wenn diese wieder $\mathrm{minsupp}$ �bersteigen, werden sie in $L_k$ aufgenommen. Der Algorithmus terminiert, wenn keine der erzeugten Kandidatenmengen in $L_k$ aufgenommen wurde, also $L_k = \emptyset$.
\begin{algorithm}
	\caption{apriori Algorithmus}
	\label{apriori}
\begin{algorithmic}
	\STATE $L_1\gets \{\text{alle Items mit } \mathrm{minsupp}\}$
	\FOR{$k = 2,\;L_{k-1} = \emptyset,\;k++$}
		\STATE $L_k \gets \text{apriori-gen}(L_{k-1})$ \COMMENT{Erzeugung von k-elementigen Kandidatenmengen}
		\FORALL{$t \in D$}
		\STATE $\text{erh�he Z�hler aller Kandidatenmengen in } C_k \text{ mit } t$
		\ENDFOR \\
		$L_k \gets \text{alle Kandidatenmengen } C_k \text{ mit } \mathrm{minsupp}$
	\ENDFOR
\end{algorithmic}
\end{algorithm}

Die Erzeugung von $k$-elementigen Kandidatenmengen verl�uft in zwei Schritten. Zun�chst werden alle $(k-1)$-elementigen Itemmengen paarweise vereinigt, die eine gemeinsame $(k-2)$-elementige Teilmenge haben. Bei den so entstandenen $k$-elementigen Kandidaten wird �berpr�ft, ob alle $(k-1)$-elementigen Teilmengen in $L_k-1$ enthalten sind, wenn nicht, dann entferne die jeweilige Menge aus $C_k$.

\begin{algorithm}
	\caption{Erzeugung der Kandidatenmengen}
	\label{apriori-gen}
\begin{algorithmic}
	\STATE \COMMENT{Erzeugung von k-elementigen Kandidatenmengen}
	\STATE $A = \{i_{j_1},\ldots, i_{j_{k-1}}\} \in L_{k-1} \text{ und } B = \{i_{l_1},\ldots, i_{l_{k-1}}\} \in L_{k-1}$
	\IF{$A \triangle B = \{i_{j_{k-1}}, i_{l_{k-1}}\}$}
		\STATE $A \cup B \text{ wird als Kandidatenmenge zu } C_k \text{ hinzugef�gt}$
	\ENDIF
	\STATE \COMMENT{Entfernen von Kandidaten, deren (k-1)-elementigen Teilmengen der Obermenge bestehen}
	\FORALL{$c \in C_k$}
		\FORALL{$(k-1)-\text{elementige Teilmengen s von } c$}
		\IF{$s \notin L_{k-1}$}
			\STATE $C_k \gets C_k \setminus c$
		\ENDIF
		\ENDFOR
	\ENDFOR
\end{algorithmic}
\end{algorithm}
\subsubsection{Problem 2: Erzeugen von Assoziationsregeln}
Nun sollen aus den gefundenen Mengen in Schritt 1 Assoziationsregel erzeugt werden. F�r jede gefundenen h�ufigen Itemmenge $H$ werden folgende Schritte ausgef�hrt.  
\begin{enumerate}
	\item Berechne alle nichtleeren Teilmengen von $H$, also $\mathcal{P}(H) \setminus \emptyset$.
	\item F�r jede Teilmenge $a \in \mathcal{P}(H) \setminus \emptyset$ erzeuge eine Regel der Form $a \implies H \setminus a$.
	\item Berechne f�r die erzeugte Regel die Konfidenz $\mathrm{conf}(a \implies H \setminus a)$
	\item  Ist $\mathrm{conf}(a \implies H \setminus a) \geq \mathrm{minconf}$, dann ist $a \implies H \setminus a$ eine Assoziationregel unter den gegebenen Bedingungen.
\end{enumerate}

Einen effizienten Algorithmus, der nach diesem Verfahren Assoziationsregeln erzeugt ist zu finden in \cite{Agrawal:1994:FAM:645920.672836}.
