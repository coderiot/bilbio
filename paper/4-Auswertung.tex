\section{Auswertung der Ergebnisse}
\subsection{Analyse der Oberthemen}
Zun�chst haben wir die Themen, die im Header jedes Publikationseintrags auftreten, die sogenannten \PY{n+nt}{<setSpec}\PY{n+nt}{>}, untersucht.
Dabei konnten wir 19 einzigartige Themen feststellen, 13 davon waren Unterthemen der Physik.
Die anderen 6 waren wie folgt: $math$ (Mathematics), $cs$ (Computer Science), $nlin$ (Nonlinear Sciences), $q-bio$ (Quantitative Biology), $stat$ (Statistics) und $q-fin$ (Quantitative Finance).
Wir haben auf dem Gesamtdatensatz den apriori-Algorithums mit den Parametern $minsupp$ = 0.5\% und $minconf$ = 50\% angewandt, und dabei die folgenden 8 Assoziationsregeln gewonnen (Tabelle \ref{tab:header}). %cite table
Weiterhin haben wir die Unterthemen der Physik zusammengefasst und dann �berpr�ft, was f�r Assoziationsregeln sich ergeben.
Die Ergebnisse sind in Tabelle \ref{tab:headerCompact} dargestellt.

\subsubsection{Assoziationsregeln}
\begin{table}[H]
\centering % used for centering table
\begin{tabular}{| r l | l | l | l |}
		\hline
		\textbf{Regel}& &\textbf{Support} &\textbf{Konfidenz} & \textbf{Lift}\\
		\hline
		\{math\}  $\implies$ & \{stat\} & 0.6\% & 64\% &3.0  \\
		\{physics:math-ph\} $\implies$& \{math\} &3.8 \% & 100\% & 4.7 \\
		\{physics:hep-th, physics:math-ph\}  $\implies$& \{math\} &0.9 \% &100\% &4.7 \\
		\{math, physics:hep-th\}  $\implies$& \{physics:math-ph\}  &0.9 \% &63\% &16.3 \\
		\{physics:gr-qc, physics:hep-th\} $\implies$& \{physics:hep-th\} &0.6 \% &72 \% &6.1 \\
		\{physics:gr-qc, physics:hep-th\} $\implies$& \{physics:astro-ph\} &0.6 \% &70 \% &3.5 \\
		\{physics:gr-qc, physics:astro-ph\} $\implies$& \{physics:hep-th\} &0.9 \% &50 \%  &4.3 \\
		\{physics:astro-ph, physics:hep-th\} $\implies$& \{physics:gr-qc\}  &0.9 \% &74 \% &12.4 \\
		\hline
\end{tabular}
\caption{Assoziationsregeln der Oberthemen minsupp=0.5\% und minconf=50\%}
\label{tab:header}
\end{table}

\begin{table}[H]
\centering % used for centering table
\begin{tabular}{|rl|l|l|l|}
	\hline
	\textbf{Regel}& &\textbf{Support} &\textbf{Konfidenz} & \textbf{Lift}\\
	\hline
	$\{\emptyset\} \implies$ & \{physics\} & 78\% &78\% &1.0  \\
	\{stat\} $\implies$ & \{math\}  &0.6 \% &63 \% &3.0 \\
	\{nlin\} $\implies$ & \{physics\}  &1.3 \% &50 \% &0.64 \\
	\{math, nlin\} $\implies$ & \{physics\}  &0.4 \% &83 \% &1.1 \\
	\hline
\end{tabular}
 \caption{Assoziationsregeln mit zusammegefassten physics-Themen: Support 0.1 \% und Konfidenz 50 \%}
\label{tab:headerCompact}
\end{table}

\subsection{Analyse der Themen in den Metadaten}
\subsubsection{Assoziationsregeln}
Wie wir bereits gesehen haben, k�nnen wir anhand der Themen in den \PY{n+nt}{<setSpec}\PY{n+nt}{>} Attributen der Header relativ wenige interessante Informationen gewinnen.
Das liegt zu einem an der relativ geringen Anzahl von Themen, die da vergeben werden, und zum anderen daran, dass diese Themen viel zu allgemein sind.
Deshalb haben wir unsere weiteren Analysen anhand der Subjects, die in den \PY{n+nt}{<dc: subject}\PY{n+nt}{>} Metadatentags auftauchen, durchgef�hrt und zwar, auf dem Oberthema $Computer Science$ beschr�nkt.

%Themen wie Math-Combinatorics tauchen auf, weil die jeweiligen Publicationen auch math in den <setSpec>s enthalten haben.

\begin{table}[H]
\centering % used for centering table
\begin{tabular}{|rl|l|l|l|}
	\hline
	\textbf{Regel}& &\textbf{sup} &\textbf{conf} &\textbf{lift}\\
	\hline
	\small I.2.7 $\implies$ &\small CS - Computation and Language &1.2\% &90 \% &16.9 \\
	\small CS- Systems and Control $\implies$ &  \small Math - Optimization and Control &1.5\% &88 \% &34 \\
	\small Math - Optimization and Control $\implies$ & \small CS - Systems and Control  &1.5\% &56 \% &34 \\
	\small CS - Social and Information Networks $\implies$ & \small Physics - Physics and Society &1.7\% &82 \% &25.5 \\
	\small Physics - Physics and Society $\implies$ & \small CS - Social and Information Networks &1.7\% &52 \% &25.5 \\
	\small F.4.1 $\implies$ & \small CS - Logic in Computer Science &1.6\% &78 \% &10.6 \\
	\small Math - Combinatorics $\implies$ & \small CS - Discrete Mathematics  &1.7\% &54 \% &7.9 \\
	\hline
\end{tabular}
 \caption{Assoziationsregeln der Unterthemen: Support 1 \% und Konfidenz 50 \%}
\end{table}
\begin{table}[H]
\centering % used for centering table
\begin{tabular}{|rl|l|l|l|}
	\hline
	\textbf{Regel}& &\textbf{sup} &\textbf{conf} &\textbf{lift}\\
	\hline
	\{D.2.9, J.1\} $\implies$ & \{H.4.1\} & $0.1\%$ &$ 100 \%$ & $473.0$\\
	\{D.2.9, H.4.1\} $\implies$ & \{J.1\} & $0.1\%$ & $100 \%$ & $500.8$\\
	\{D.2.9, J.1\} $\implies$ & \{K.6.4\} & $0.1\%$ & $100 \%$ & $479.6$\\
	\{J.1, K.6.4\} $\implies$ & \{D.2.9\} & $0.1\%$ & $100 \%$ & $577.2$\\
	\{D.2.9, H.4.1\} $\implies$ & \{K.6.4\} &$0.1\%$ & $100 \%$ & $479.6$\\
	\{J.1, K.6.4\} $\implies$ & \{H.4.1\} & $0.1\%$ & $100 \%$ & $473.0$\\
	\{H.4.1, K.8.1\} $\implies$ & \{K.6.4\} &$ 0.2\%$ & $100 \%$ & $479.6$\\
	\{K.6.4, K.8.1\} $\implies$ & \{H.4.1\} &$ 0.2\%$ & $100 \%$ & $473.0$\\
	\{D.2.5,K.8.1\} $\implies$ & \{H.4.1\}  &$ 0.1\%$ & $100 \%$ & $473.0$\\
	\{D.2.5, H.4.1\} $\implies$ & \{K.8.1\} &$0.1\%$ & $100 \%$ & $567.6$\\
	\hline
\end{tabular}
 \caption{Assoziationsregeln der Unterthemen sortiert nach Konfidenz mit:  Support 0.1 \% und Konfidenz 50 \%}
\end{table}
\begin{table}[H]
\centering % used for centering table
\begin{tabular}{|rl|l|l|l|}
	\hline
	\textbf{Regel}& &\textbf{sup} &\textbf{conf} &\textbf{lift}\\
	\hline
	\{J.1, K.6.4\} $\implies$ &\{D.2.9\} &0.1\% & 100 \% & 577.2 \\
	\{H.4.1, J.1, K.6.4\} $\implies$ & \{D.2.9\} &0.1\%& 100\% & 577.2\\
	\{D.2.5, H.4.1\} $\implies$ & \{K.8.1\} &0.1\%& 100\% & 567.6\\
	\{D.2.5, K.6.4\} $\implies$ & \{K.8.1\} &0.1\%& 100\% & 567.6\\
	\{D.2.5, H.4.1, K.6.4\} $\implies$ & \{K.8.1\} &0.1\%& 100 \% & 567.6\\
	\{H.4.1, J.1\} $\implies$ & \{D.2.9\} &0.1\%& 92.7 \% & 534.9\\
	\{D.2.9, H.4.1\} $\implies$ & \{J.1\} &0.1\%& 100 \% & 500.8\\
	\{D.2.9, H.4.1, K.6.4\} $\implies$ & \{J.1\} &0.1\%& 100\% & 500.8\\
	\{D.2.9, K.6.4\} $\implies$ & \{J.1\} &0.1\%& 97.4\% & 487.9\\
	\{D.2.9, J.1\} $\implies$ & \{K.6.4\} &0.1\%& 100\% & 479.6\\
	\hline
\end{tabular}
 \caption{Assoziationsregeln der Unterthemen sortiert nach Lift mit:  Support 0.1 \% und Konfidenz 50 \%}
\end{table}
\begin{table}[H]
\begin{tabular}{|rl|l|l|l|}
	\hline
	\textbf{Regel} & &\textbf{sup} &\textbf{conf} &\textbf{lift}\\
	\hline
	\small Math - Combinatorics $\implies$ & \small CS - Discrete Mathematics  &1.7\% &54 \% &7.9 \\
	\small CS - Social and Information Networks $\implies$ & \small Physics - Physics and Society &1.7\% &82 \% &25.5 \\
	\small Physics - Physics and Society $\implies$ & \small CS - Social and Information Networks &1.7\% &52 \% &25.5 \\
	\small F.4.1 $\implies$ & \small CS - Logic in Computer Science &1.6\% &78 \% &10.6 \\
	\small CS- Systems and Control $\implies$ & \small Math - Optimization and Control &1.5\% &88 \% &34 \\
	\small I.2.7 $\implies$ & \small CS - Computation and Language &1.2\% &90 \% &16.9 \\
	\small Math - Optimization and Control $\implies$ & \small CS - Systems and Control  &1.5\% &56 \% &34 \\
	\small I.2.7(Natural Language Processing)  $\implies$ & \small CS - Computation and Language  &1.2\% &91 \% &16.9 \\
	\small G.2.2(Graph Theory)   $\implies$ & \small CS - Discrete Mathematics  &0.8\% &59 \% &8.8 \\
	\small F.1.3(Complexity Measures)  $\implies$ & \small CS - Computational Complexity  &0.7\% &86 \% &12.6 \\
	\hline
\end{tabular}
 \caption{Assoziationsregeln der Unterthemen sortiert nach Support mit:  Support 0.1 \% und Konfidenz 50 \%}
\end{table}
\subsubsection{Mappings zwischen Klassifikationen}
Die in der folgenden Tabelle dargestellten Regeln sind alle Assoziationsregeln, wo genau ein ACM-Thema auf der linken Seite und ein arxiv-Thema auf der rechten Seite steht.
Wir haben explizit nach Regeln, die diese Form haben, gesucht, damit eine eindeutige Abbildung zwischen Subjects in der jeweiligen Klassifikationen m�glich ist.
Wir haben keine Regeln der Form $arxiv \rightarrow ACM$ gefunden, was auch logisch ist, da die ACM-Klassifikation eine viel feinere Unterteilung als arxiv anbietet.
Alle gefundenen Regeln haben relativ hohe Konfidenz- und Liftwerte, was daf�r spricht, dass die so entstandenen Mappings zuverl�ssig sind.

\begin{table}[H]
\centering % used for centering table
\begin{tabular}{|rl|c|c|c|}
	\hline
	\textbf{Regel(ACM $\implies$ }  &\textbf{arXiv.org)} & {Support} &\textbf{Konfidenz} & \textbf{Lift}\\
	\hline
	\small G.4(Mathematical Software) $\implies$ & \small CS - Mathematical Software & 0.1\% &51\% &67.4  \\
	\small K.4.m(Computers and Society) $\implies$ & \small CS - Computers and Society &0.2 \% &96 \%  &53.4 \\
	\small I.2.9(Robotics) $\implies$ & \small CS - Robotics &0.1 \% & 75 \%  &52.1 \\
	\small H.3.7(Digital Libraries) $\implies$ & \small CS - Digital Libraries &0.3 \% &89 \% &48.7 \\
	\small I.1.2(Symbolic Manipulation - Algorithm) $\implies$ & \small CS - Symbolic Computation  &0.1 \% &60 \%  &41.3 \\
	\small I.2.11(Distributed Artificial Intelligence) $\implies$ & \small CS - Multiagent Systems  &0.2 \% &53 \%  &35.9 \\
	\small H.5.2(User Interfaces) $\implies$ & \small CS - Human-Computer Interaction &0.2 \% &58 \%  &33.7 \\
	\small I.3.5(Computational Geometry) $\implies$ & \small CS - Computational Geometry &0.3 \% &88 \%  &31.2 \\
	\small H.2.3(Datebase Managment - Languages) $\implies$ & \small CS - Databases &0.2 \% &91 \%  &30.6 \\
	\hline
\end{tabular}
 \caption{Assoziationsregeln der Unterthemen: Support 0.1 \% und Konfidenz 50 \%}
\end{table}
\subsubsection{Analyse �ber die Zeit}
