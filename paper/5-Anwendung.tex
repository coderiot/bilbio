\section{Further Research und m�gliche Anwendung}

%further research:
%normalisierte zeitliche analyse; vergleich zum ganzen
%kombination von themen und key words im titel/abstract bei der Assoziationsanalyse
%
%anwendungen:
%mappings
%tag / subject vorschlag systeme
%recommender systeme f�r papers

\subsection{Weiterf�hrende Untersuchungen}
Wir haben im Rahmen unseres Projekts viele bedeutende Zusammenh�nge aufgedeckt.
Allerdings gibt es noch ein paar interessante Aspekte, denen wir, vor allem aufgrund zeitlichen Mangel, nicht nachgegangen sind.
F�rs Erste w�re es wichtig, die zeitliche Entwicklung aller im Datensatz vorkommenden Themen zu untersuchen.
Die zeitliche Analyse der Themen innerhalb der Informatik wird fundierter und aussagekr�ftiger, wenn sie in Relation zur Entwicklung der gesamten Datenbank gesetzt wird.
In dem Fall werden wir imstande sein, bestimmte Tendenzen, wie z.B. die niedrige Anzahl an Publikationen vor dem Jahr 2008, besser zu begr�nden.
Eine weitere spannende Untersuchung w�re, andere Faktoren in unsere Analyse miteinzubeziehen.
Man k�nnte studieren, was f�r Assoziationsregeln sich ergeben, wenn nicht nur die vergebenen Themen, sondern auch Schl�sselw�rter aus dem Titel oder Abstract oder die Autoren betrachtet werden.
So k�nnen zus�tzliche Erkenntnisse gewonnen werden, z.B. welche Autoren in welchen Themengebieten besonders wichtig sind, oder was f�r Begriffe mit einem Thema assoziiert werden.

\subsection{Anwendungen}
Die im Projekt gefundenen Assoziationsregeln k�nnen auf mehreren Weisen verwertet werden.
Eine m�gliche Anwendung - die Erstellung von Mappings zwischen verschiedenen standartisierten Klassifikationen, wurde bereits im Kapitel 4 angedeutet.
Diese k�nnen beim Hochladen von Publikationen in verschiedene wissenschaftliche Datenbanken sehr n�tzlich sein.
Wenn sinnvolle und zuverl�ssliche Abbildungen zwischen verschiedenen Klassifikationen ermittelt werden, k�nnen alle f�r ein Paper in einer Notation vergebene Themen automatisch in eine andere �berf�hrt werden.
Ferner k�nnen solche Mappings und Assoziationsregeln generell f�r das Entwickeln eines Subject-Vorschlagsystems verwendet werden.
Ein solches System wird in der Lage sein, einem Nutzer, der seine Arbeit online stellt und mit Themen kennzeichnet, geeignete Vorschl�ge f�r weitere Themen, die relevant sein k�nnten, oder ihre entsprechenden Formulierungen in einer anderen Klassifikation zu machen.
Eine weitere Anwendung von Assoziationsregeln kann im Entwurf von Paperrecommendersystemen gefunden werden.
Zweck dieser Systeme ist es, einer Leserin, die ein Paper interessant gefunden hat, weitere relevante/verwandte Arbeiten anzubieten.
F�r das Gestalten so eines Systems werden auch genau solche Assoziationsregeln relevant, die nicht so oft auftreten (einen niedrigen $\mathrm{support}$ haben), daf�r aber relativ stark sind (h�here $\mathrm{confidence}$ und $\mathrm{lift}$ Werte haben).
Dann werden der Nutzerin eine geringe Anzahl an Publikationen mit hoher Relevanz angeboten.
