\section{Zusammenfassung}
Wie in dieser Ausarbeitung gezeigt, ist die Assoziationsanalyse der Themenwissenschaftlichen Publikationen eine wichtige Methode der bibliometrischen Untersuchungen.
Durch die Betrachtung der vergebenen Subjects f�r den arxiv-Datensatz haben wir bedeutende Erkenntnisse �ber die Verwandschaft manchen Themen als auch �ber die Entwicklung �ber die Zeit von bestimmten Themengebieten gewonnen.
Anlehnend and die Ergebnisse, die dieses Paper pr�sentiert, k�nnen standartisierten Mappings zwischen verschiedenen Klassifikationen, Themenvorschlag- sowie Paperrecommendersysteme entwickelt werden.
Um diese erfolgreich umzusetzen, bietet sich an, neben der Assoziationsanalyse von Themen auch weitere Methoden, wie z.B. Autorenanalyse oder Untersuchung der Schl�sselw�rter des Titels und/oder des Abstracts, zu verwenden.

%* was gewinnen wir durch Assoziationsanalyse von Themen
%* was haben wir gemacht
%* hinweis auf weiterf�hrende recherche

%In a general way,
%
%restate your topic and why it is important,
%restate your thesis/claim,
%address opposing viewpoints and explain why readers should align with your position,
%call for action or overview future research possibilities.
%
%from my specific topic to more general matters
%say what you have already said but do it quickly, sharply and in different words
