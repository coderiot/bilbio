\documentclass{lni}


%\IfFileExists{latin1.sty}{\usepackage{latin1}}{\usepackage{isolatin1}}
\usepackage[latin1]{inputenc}
% Verwenden von T1 Fonts
\usepackage[T1]{fontenc}
\usepackage[ngerman]{babel}
\usepackage{graphicx}
\usepackage{amsmath}
\usepackage{amssymb}
\usepackage{listings}
%\usepackage{hyperref}
\usepackage{fancyvrb}
\usepackage{color}
\usepackage{algorithmic}
\usepackage{algorithm}
\usepackage{float}

\floatstyle{ruled}
\newfloat{program}{H}{lop}
\floatname{program}{Listing}
\include{source/syntax}
\usepackage{lastpage}
\thispagestyle{plain}
\pagenumbering{arabic}


\author{
	peterrr, Lusy \\
	\\
	%Abteilung \\
	Freie Universit�t Berlin \\
	%Anschrift \\
	%Postleitzahl Ort \\
	emaiaddresse@autor1 \\
	emaiaddresse@autor2
}
\title{Auffinden von Assoziationsregeln zwischen Themen in wissenschaftlichen Publikationen}

\newtheorem{mdef}{Definition}
\begin{document}
\maketitle

\begin{abstract}
Mit Hilfe der Assoziationsanalyse sollen Beziehungen zwischen Themen von wissenschaftlichen Publikationen gefunden werden. Hierf�r wird auf den aprior-Algorithmus zur�ckgegriffen. Der Analyse zugrundeliegende Datensatz waren alle Publikationen von arxiv.org mit dem Stand Oktober 2011. Die gefundenen Beziehungen k�nnen, unter anderem, f�r Erstellung von Recommendersystemen oder Mappings zwischen verschiedenen wissenschaftlichen Klassifikationen genutzt werden.
\end{abstract}

\tableofcontents
\newpage
%-------------------------------------------------------
%----Hauptteil------------------------------------------
%-------------------------------------------------------

\section{Einleitung}
\subsection{Motivation}
In Anlehnung an die Warenkorbeanalyse sollen die vergebenen Themen f�r wissenschaftliche Publikationen analysiert werden,
um so h�ufig miteinander vergebene Themen und Br�ckenthemen zwischen Fachgebieten zu ermitteln.
\subsection{Problemstellung}
Die Themen zu wissenschaftlichen Publikationen werden anhand der ver�ffentlichen Publikationen auf arXiv.org beispielhaft untersucht. Das Verfahren, um Beziehungen zwischen Themen zu finden, wird die Assoziationsanalyse sein. Durch Assoziationsregeln k�nnen 


\section{Assozationsanalyse}
\label{sec:assoziationsanalyse}
%\cite{Agrawal:1993:MAR:170035.170072}% TODO test um unser Startpaper zu zitieren
Die Assoziationsanalyse \cite{Tan:2005:IDM:1095618} beschreibt die Suche nach Assoziationsregeln innnerhalb einer Menge von Transaktionen. Transaktionen sind endliche Teilmengen aus einer sogenannten Itemmenge. Items sind Objekte, die in Transaktionen auftreten k�nnen. Ein Datensatz zur Assoziationsanalyse besteht also aus einer Menge von Transaktionen.
\subsection{Assozationsregeln}
Eine Assoziationsregel ist eine Implikation der Form $A \rightarrow B$ mit $A \cap B = \emptyset$, also $A$ und $B$ sind disjunkt. Die Regel $A \rightarrow B$ soll aussagen, dass wenn $A$ in einer Transaktion auftritt auch $B$ mit einer gewissen Wahrscheinlichkeit vorkommt. Um diese Wahrscheinlichkeit zu beschreiben werden die Werte des Supports und der Konfidenz verwendet.
\subsection{Kenngr��en von Assoziationsregeln}
\subsubsection{Support}
Der Support bezeichnet die relative H�ufigkeit des Auftretens der Itemmenge $X$ in einem Datensatz. Er wird unter anderem daf�r verwendet um uninteressante Assoziationsregeln herauszufiltern. Wird der minimale Support zu klein gew�hlt kann es vorkommen, dass eine Regel zuf�llig und damit nicht aussagekr�ftig ist.
\begin{mdef}[Support]
\[
\mathrm{support}(X) = \frac{\text{Anzahl der Vorkommen von X im Datensatz}}{\text{Anzahl aller Transaktionen}}
\]
\end{mdef}
\subsubsection{Konfidenz}
Die Konfidenz ist die relative H�ufigkeit der Transaktionen in denen X zusammen mit Y vorkommt. Oder anders formuliert, beschreibt diese, wie oft Y in der Datenbasis vorkommt, gegeben dass X vorkommt. Die Konfidenz kann also als Ma� der Zuverl��igkeit f�r die Aussage einer Regel angesehen werden.
\begin{mdef}[Konfidenz]
\[
\mathrm{conf}(X\rightarrow Y) = \frac{\mathrm{support}(X \cup Y)}{\mathrm{support}(X)}
\]
\end{mdef}
\subsubsection{Lift}
Regeln mit einem hohen Konfidenzwert k�nnen irref�hrend sein, da der der Support f�r die rechte Seite der Regel nicht mit einbezogen wird. Deswegen wird der Lift f�r die St�rke einer Regel bzw. deren Bedeutung herrangezogen.
Der Lift berechnet den Anteil der Konfidenz der Regel und dem Support der rechten Seite der Regel und ist ein Faktor wieviel �fter $X$ und $Y$ zusammen in der Datenbasis auftreten, als getrennt.
\begin{mdef}[Lift]
\[
\mathrm{lift}(X\rightarrow Y) =\frac{\mathrm{conf}(X \rightarrow Y)}{\mathrm{support}(Y)} =  \frac{ \mathrm{support}(X \cup Y)}{ \mathrm{support}(Y) \cdot \mathrm{support}(X) }
\]
\end{mdef}
\subsection{apriori-Algorithmus}
Der apriori-Algorithmus wurde 1994 von \cite{Agrawal:1994:FAM:645920.672836}  ver�ffentlicht.
Dieser unterscheidet sich fundamental zu denbis dato existierenden Algorithmen zum Finden von Assoziationsregeln.
Der Algorithmus baut auf folgende Sachverhalte auf:
Sei $I = \{i_1, i_2, \ldots i_m\}$ eine endliche Menge von Items, die in einer Transaktion vorkommen k�nnen.
$D = \{t_1, t_2, \ldots , t_n\}$ ist eine Menge von Transaktionen, wobei jede Transaktion $T \subseteq I$ eine Teilmenge der Menge von Items ist.
Weiterhin werden ein minimaler Support $\mathrm{minsupp}$ und eine minimale Konfidenz $\mathrm{minconf}$ definiert, die als Einschr�nkungskriterien f�r die gesuchten Assoziationsregeln verwendet werden.
Der Algorithmus wird in zwei Probleme aufgeteilt.
\begin{description}
	\item[Problem 1] Finden aller Teilmengen von Items in den Transaktionen, die mindestens einen Support von $\mathrm{minsupp}$ haben.
	\item[Problem 2] Aus den gefundenen Teilmengen in Problem 1 werden Assoziationsregeln erzeugt.
    Nur Regeln, die mindestens eine Konfidenz von $\mathrm{minconf}$ haben, werden ber�cksichtigt.
\end{description}
\subsubsection{Problem 1: Finden von h�ufigen Mengen}
Sei $L_k$ die Menge aller Mengen mit k Items, die mindestens einen Support von $\mathrm{minsupp}$ haben, und $C_k$ die Menge m�glicher Kandidaten f�r $L_k$.
Im ersten Schritt wird die H�ufigkeit aller Items in den Transaktionen aus dem gesamten Datensatz $D$ berechnet und die Items, die mindestens $\mathrm{minsupp}$ haben, bilden die Menge $L_1$.
In der k-ten Iteration werden neue Kandidaten $C_k$ erzeugt und wenn diese wieder $\mathrm{minsupp}$ �bersteigen, werden sie in $L_k$ aufgenommen.
Der Algorithmus terminiert, wenn keine der gefundenen Kandidatenmengen in $L_k$ aufgenommen wurde, also $L_k = \emptyset$.
\begin{algorithm}[H]
	\caption{apriori Algorithmus}
	\label{apriori}
\begin{algorithmic}[i]
	\STATE $L_1\gets \{\text{alle Items mit } \mathrm{minsupp}\}$
	\FOR{$k = 2,\;L_{k-1} = \emptyset,\;k++$}
		\STATE $L_k \gets \text{apriori-gen}(L_{k-1})$ \COMMENT{Erzeugung von k-elementigen Kandidatenmengen}
		\FORALL{$t \in D$}
		\STATE $\text{erh�he Z�hler aller Kandidatenmengen in } C_k \text{ mit } t$
		\ENDFOR \\
		$L_k \gets \text{alle Kandidatenmengen } C_k \text{ mit } \mathrm{minsupp}$
	\ENDFOR
\end{algorithmic}
\end{algorithm}
Die Erzeugung von $k$-elementigen Kandidatenmengen verl�uft in zwei Schritten.
Zun�chst werden alle $(k-1)$-elementigen Itemmengen paarweise vereinigt, die eine gemeinsame $(k-2)$-elementige Teilmenge haben.
Bei den so entstandenen $k$-elementigen Kandidaten wird �berpr�ft, ob alle $(k-1)$-elementigen Teilmengen in $L_{k-1}$ enthalten sind, wenn nicht, dann entferne die jeweilige Menge aus $C_k$.
\begin{algorithm}[H]
	\caption{Erzeugung der Kandidatenmengen}
	\label{apriori-gen}
\begin{algorithmic}[i]
	\STATE \COMMENT{Erzeugung von k-elementigen Kandidatenmengen}
	\STATE $A = \{i_{j_1},\ldots, i_{j_{k-1}}\} \in L_{k-1} \text{ und } B = \{i_{l_1},\ldots, i_{l_{k-1}}\} \in L_{k-1}$
	\IF{$A \triangle B = \{i_{j_{k-1}}, i_{l_{k-1}}\}$}
		\STATE $A \cup B \text{ wird als Kandidatenmenge zu } C_k \text{ hinzugef�gt}$
	\ENDIF
	\STATE \COMMENT{Entfernen von Kandidaten, deren (k-1)-elementigen Teilmengen der Obermenge bestehen}
	\FORALL{$c \in C_k$}
		\FORALL{$(k-1)-\text{elementige Teilmengen s von } c$}
		\IF{$s \notin L_{k-1}$}
			\STATE $C_k \gets C_k \setminus c$
		\ENDIF
		\ENDFOR
	\ENDFOR
\end{algorithmic}
\end{algorithm}
\subsubsection{Problem 2: Erzeugen von Assoziationsregeln}
Nun sollen aus den gefundenen Mengen in Problem 1 Assoziationsregeln erzeugt werden.
F�r jede gefundene h�ufige Itemmenge $H$ werden folgende Schritte ausgef�hrt:
\begin{enumerate}
	\item Berechne alle nichtleeren Teilmengen von $H$, also $\mathcal{P}(H) \setminus \emptyset$.
	\item F�r jede Teilmenge $a \in \mathcal{P}(H) \setminus \emptyset$ erzeuge eine Regel der Form $a \implies H \setminus a$.
	\item Berechne f�r die erzeugte Regel die Konfidenz $\mathrm{conf}(a \implies H \setminus a)$
	\item  Ist $\mathrm{conf}(a \implies H \setminus a) \geq \mathrm{minconf}$, dann ist $a \implies H \setminus a$ eine Assoziationsregel unter den gegebenen Bedingungen.
\end{enumerate}

Ein effizienter Algorithmus, der nach diesem Verfahren Assoziationsregeln erzeugt, ist in \cite{Agrawal:1994:FAM:645920.672836} zu finden.

\section{Der Datensatz}
arXiv.org ist ein Dokumentenserver f�r Preprints aus dem Bereich der naturwissenschaftlichen F�cher.
Ver�ffentlichungen k�nnen ohne Begutachtung auf dem Server abgelegt werden und werden nach dem Open-Access-Prinzip f�r
alle kostenfrei einsehbar.

Der von uns verwendete Datensatz liegt im XML-Format vor und benutzt ein von der Dublin Core Initiative \footnote{http://dublincore.org/} spezifiziertes Schema.
Jeder Eintrag f�r eine Publikation wird von einem \PY{n+nt}{<record}\PY{n+nt}{>}-Tag umschlossen und alle Eintr�ge werden in \PY{n+nt}{<ListRecords}\PY{n+nt}{>} aufgef�hrt.
\begin{Verbatim}[commandchars=\\\{\}, fontsize=\footnotesize, frame=single]
\PY{c+cp}{<?xml version="1.0" encoding="UTF-8"?>}
\PY{n+nt}{<OAI-PMH} \PY{n+na}{xmlns=}\PY{l+s}{"http://www.openarchives.org/OAI/2.0/"}
  \PY{n+na}{xmlns:xsi=}\PY{l+s}{"http://www.w3.org/2001/XMLSchema-instance"} 
  \PY{n+na}{xsi:schemaLocation=}\PY{l+s}{"http://www.openarchives.org/OAI/2.0/OAI-PMH.xsd"}\PY{n+nt}{>}
  \PY{n+nt}{<responseDate}\PY{n+nt}{>}
    2011-10-05T16:49:31Z
  \PY{n+nt}{</responseDate>}
  \PY{n+nt}{<request} \PY{n+na}{verb=}\PY{l+s}{"ListRecords"} \PY{n+na}{metadataPrefix=}\PY{l+s}{"oai\PYZus{}dc"}\PY{n+nt}{>}
    http://export.arxiv.org/oai2
  \PY{n+nt}{</request>}
    \PY{n+nt}{<ListRecords}\PY{n+nt}{>}
      \PY{n+nt}{<record}\PY{n+nt}{>}
        \PY{n+nt}{<header}\PY{n+nt}{>}
          ...
        \PY{n+nt}{</header>}
        \PY{n+nt}{<metadata}\PY{n+nt}{>}
          ...
        \PY{n+nt}{</metadata>}
      \PY{n+nt}{</record>}
        ...
      \PY{n+nt}{<record}\PY{n+nt}{>}
        \PY{n+nt}{<header}\PY{n+nt}{>}
          ...
        \PY{n+nt}{</header>}
        \PY{n+nt}{<metadata}\PY{n+nt}{>}
          ...
        \PY{n+nt}{</metadata>}
      \PY{n+nt}{</record>}
    \PY{n+nt}{</ListRecords>}
\PY{n+nt}{</OAI-PMH>}
\end{Verbatim}

\subsection{Aufbau der Eintr�ge}
Die Eintr�ge setzen sich aus zwei Teilen zusammen: den \PY{n+nt}{<header}\PY{n+nt}{>}- und \PY{n+nt}{<metadata}\PY{n+nt}{>}-Teil.
\subsubsection{Header}
Im Header-Teil eines Eintrages sind arXiv-spezifische Informationen zur jeder Publikation zu finden.
Der <identifier> ist ein eindeutiger Bezeichner f�r eine Publikation in der arXiv-Datenbank.
Das <datestamp>-Attribut enth�lt das Datum der letzten Bearbeitung des Eintrags auf dem arXiv-Server.
Die <setSpec>-Attribute enthalten die von arXiv zugeordneten Themen f�r eine Publikation.
\begin{Verbatim}[commandchars=\\\{\}, fontsize=\footnotesize, frame=single]
\PY{n+nt}{<header}\PY{n+nt}{>}
  \PY{n+nt}{<identifier}\PY{n+nt}{>}oai:arXiv.org:0704.0002\PY{n+nt}{</identifier>}
  \PY{n+nt}{<datestamp}\PY{n+nt}{>}2008-12-13\PY{n+nt}{</datestamp>}
  \PY{n+nt}{<setSpec}\PY{n+nt}{>}cs\PY{n+nt}{</setSpec>}
  \PY{n+nt}{<setSpec}\PY{n+nt}{>}math\PY{n+nt}{</setSpec>}
\PY{n+nt}{</header>}
\end{Verbatim}

\subsubsection{Metadaten}
In den Metadaten befinden sich alle anderen relevanten Informationen zu einer Publikation. Dazu geh�ren z.B. der Titel der Publikation,
die Autoren, sowie die Themen. Themen werden anhand unterschiedlicher Klassifizierungen vergeben. Es werden sowohl arXiv-spezifische Themen vergeben als auch Themen, die auf standartisierten Klassifizierungen, zum Beispiel der ACM-Klassifizierung\footnote{http://www.acm.org/about/class/ccs98-html} oder der MSC-Klassifizierung\footnote{http://msc2010.org/mscwiki/index.php?title=MSC2010}, beruhen. Zu jedem Eintrag einer Publikation geh�rt au�erdem ein kurzer Abstract, der den Inhalt kurz erl�utern soll. Anschlie�end gibt es ein oder mehrere <date>-Attribute, die das Datum aller Ver�nderungen an dem Eintrag der Publikation aufzeichnen. Am Ende befinden sich verschiedene eindeutige Bezeichner, um die Publikation in verschiedenen Datenbanken wieder zu finden.
\begin{Verbatim}[commandchars=\\\{\}, fontsize=\tiny, frame=single]
\PY{n+nt}{<dc:title}\PY{n+nt}{>}Titel des Papers\PY{n+nt}{</dc:title>}
\PY{n+nt}{<dc:creator}\PY{n+nt}{>}Author 1\PY{n+nt}{</dc:creator>}
\PY{n+nt}{<dc:creator}\PY{n+nt}{>}Author 2\PY{n+nt}{</dc:creator>}
\PY{n+nt}{<dc:subject}\PY{n+nt}{>}\textcolor{red}{\bf{Physics - Optics}}\PY{n+nt}{</dc:subject>}
\PY{n+nt}{<dc:subject}\PY{n+nt}{>}\textcolor{red}{\bf{Mathematics - Combinatorics}}\PY{n+nt}{</dc:subject>}
\PY{n+nt}{<dc:description}\PY{n+nt}{>}Description\PY{n+nt}{</dc:description>}
\PY{n+nt}{<dc:description}\PY{n+nt}{>}Comment\PY{n+nt}{</dc:description>}
\PY{n+nt}{<dc:date}\PY{n+nt}{>}\textcolor{red}{\bf{2007-04-02}}\PY{n+nt}{</dc:date>}
\PY{n+nt}{<dc:date}\PY{n+nt}{>}\textcolor{red}{\bf{2007-07-24}}\PY{n+nt}{</dc:date>}
\PY{n+nt}{<dc:type}\PY{n+nt}{>}text\PY{n+nt}{</dc:type>}
\PY{n+nt}{<dc:identifier}\PY{n+nt}{>}http://arxiv.org/abs/0704.0001\PY{n+nt}{</dc:identifier>}
\PY{n+nt}{<dc:identifier}\PY{n+nt}{>}Phys.Rev.D76:013009,2007\PY{n+nt}{</dc:identifier>}
\end{Verbatim}


\subsection{Parsen des Datensatzes?}

\subsection{Eigenschaften}
Der Datensatz besteht aus $706\,077$ Eintr�gen mit wissenschaftlichen Publikationen. Die Publikationen werden im Header-Teil eines Eintrags in 18 Themen(hier Oberthemen genannt) eingeteilt. Eine Publikatione kann ein oder mehrere Themen haben. In der n�chsten Tabelle ist eine Verteilung der Themen aufgelistet.\\
\begin{table}[ht]
\caption{Aufteilung der Oberthemen im  Datensatz} % title of Table
\centering % used for centering table
\begin{tabular}{| l | l | l |}
	\hline
	Thema & Abk�rzung & Anteil(ca.) \\
	\hline 
	Mathematics & math & $21.2 \%$ \\
	Condensed Matter & physics:cond-mat & $20.0 \%$\\
	Astrophysics & physics:astro-ph & $20.0\%$\\
	High Energy Physics - Phenomenology & physics:hep-ph & $13.0\%$\\
	High Energy Physics - Theory & physics:hep-th & $11.7\%$\\
	Physics & physics:physics & $6.8\%$\\
	Quantum Physics & physics:quant-ph &$6.3\%$\\
	General Relativity and Quantum Cosmology & physics:gr-qc & $6.0 \%$\\
	Computer Science & cs & $4.8 \%$ \\
	Mathematical Physics & physics:math-ph & $3.9 \%$\\
	Nuclear Experiment & physics:nucl-th & $3.8\%$\\
	High Energy Physics - Experiment & physics:hep-ex & $2.7\%$\\
	Nonlinear Sciences & nlin &  $2.7 \%$\\
	High Energy Physics - Lattice & physics:hep-lat & $2.1\%$\\
	Quantitative Biology & q-bio &$1.4 \%$ \\
	Nuclear Theory & physics:nucl-ex & $1.3\%$\\
	Statistics & stat & $1.0 \%$ \\
	Quantitative Finance & q-fin & $0.4\%$ \\
	Physics & physics & $0.0 \%$\\
	\hline
\end{tabular}
\end{table}

Die durschnittliche Anzahl der vergebenen Themen betr�gt $1.3$ und die maximale Anzahl von Themen f�r eine Publikation ist $9$. Bei rund $80 \%$ aller Publikationen wurde nur $1$ Thema im Kopfbereich eines Eintrages vergeben. Bei $14 \%$ aller Themen wurden 2 und bei den restlichen $6 \%$ mehr als 2 Themen vergeben.

\subsection{Eigenschaften vom Thema Computer Science}
Zus�tzlich sollen alle Eintr�ge mit cs als Oberthema betrachtet werden und deren Themen im metadata-Teil untersucht werden. Die Anzahl der wissenschaftlichen Publikationen mit dem Oberthema cs(Computer Science) betr�gt rund $34\;000$. Die Bezeichnungen der Themen teil sich in 3 gro�e Themengruppen auf. 
\begin{figure}[ht]
	\centering
	\includegraphics[scale=0.45]{../visual/csFrequent_filter_acm_and_msc.png}
	\caption{Fachbereich Informatik}
	%\label{fbi-logo}
\end{figure}


\section{Auswertung der Ergebnisse}
\subsection{Analyse der Oberthemen}
\subsubsection{Assoziationsregeln}
\begin{table}[H]
\centering % used for centering table
\begin{tabular}{| r l | l | l | l |}
		\hline
		\textbf{Regel}& &\textbf{Support} &\textbf{Konfidenz} & \textbf{Lift}\\
		\hline
		\{math\}  $\implies$ & \{stat\} & 0.6\% & 64\% &3.0  \\
		\{hysics:math-ph\} $\implies$& \{math\} &3.8 \% & 100\% & 4.7 \\
		\{physics:hep-th, physics:math-ph\}  $\implies$& \{math\} &0.9 \% &100\% &4.7 \\
		\{math, physics:hep-th\}  $\implies$& \{physics:math-ph\}  &0.9 \% &63\% &16.3 \\
		\{physics:gr-qc, physics:hep-th\} $\implies$& \{physics:hep-th\} &0.6 \% &72 \% &6.1 \\
		\{physics:gr-qc, physics:hep-th\} $\implies$& \{physics:astro-ph\} &0.6 \% &70 \% &3.5 \\
		\{physics:gr-qc, physics:astro-ph\} $\implies$& \{physics:hep-th\} &0.9 \% &50 \%  &4.3 \\
		\{physics:astro-ph, physics:hep-th\} $\implies$& \{physics:gr-qc\}  &0.9 \% &74 \% &12.4 \\
		\hline
\end{tabular}
\caption{Assoziationsregeln der Oberthemen minsupp=0.5\% und minconf=50\%}
\end{table}

\begin{table}[H]
\centering % used for centering table
\begin{tabular}{|rl|l|l|l|}
	\hline
	\textbf{Regel}& &\textbf{Support} &\textbf{Konfidenz} & \textbf{Lift}\\
	\hline
	$\{\emptyset\} \implies$ & \{physics\} & 78\% &78\% &1.0  \\
	\{stat\} $\implies$ & \{math\}  &0.6 \% &63 \% &3.0 \\
	\{nlin\} $\implies$ & \{physics\}  &1.3 \% &50 \% &0.64 \\
	\{math, nlin\} $\implies$ & \{physics\}  &0.4 \% &83 \% &1.1 \\
	\hline
\end{tabular}
 \caption{Assoziationsregeln mit zusammegefassten physics-Themen Support: 0.1 \% und Konfidenz 50 \%}
\end{table}
\subsection{Analyse der Themen in den Metadaten}
\subsubsection{Assoziationsregeln}
\begin{table}[H]
\centering % used for centering table
\begin{tabular}{|rl|l|l|l|}
	\hline
	\textbf{Regel}& &\textbf{sup} &section\textbf{conf} &\textbf{lift}\\
	\hline
	\small I.2.7 $\implies$ &\small CS - Computation and Language &1.2\% &90 \% &16.9 \\
	\small CS- Systems and Control $\implies$ &  \small Math - Optimization and Control &1.5\% &88 \% &34 \\
	\small Math - Optimization and Control $\implies$ & \small CS - Systems and Control  &1.5\% &56 \% &34 \\
	\small CS - Social and Information Networks $\implies$ & \small Physics - Physicsysics and Society &1.7\% &82 \% &25.5 \\
	\small Physicsics - Physics and Society $\implies$ & \small CS - Social and Information Networks &1.7\% &52 \% &25.5 \\
	\small F.4.1 $\implies$ & \small CS - Logic impliesn Computer Science &1.6\% &78 \% &10.6 \\
	\small Math - Combinatorics $\implies$ & \small CS - Discrete Mathematics  &1.7\% &54 \% &7.9 \\
	\hline
\end{tabular}
 \caption{Assoziationsregeln der Unterthemen Support: 1 \% und Konfidenz 50 \%}
\end{table}
\begin{table}[H]
\centering % used for centering table
\begin{tabular}{|rl|l|l|l|}
	\hline
	\textbf{Regel}& &\textbf{sup} &section\textbf{conf} &\textbf{lift}\\
	\hline
	\{D.2.9, J.1\} $\implies$ & \{H.4.1\} & $0.1\%$ &$ 100 \%$ & $473.0$\\
	\{D.2.9, H.4.1\} $\implies$ & \{J.1\} & $0.1\%$ & $100 \%$ & $500.8$\\
	\{D.2.9, J.1\} $\implies$ & \{K.6.4\} & $0.1\%$ & $100 \%$ & $479.6$\\
	\{J.1, K.6.4\} $\implies$ & \{D.2.9\} & $0.1\%$ & $100 \%$ & $577.2$\\
	\{D.2.9, H.4.1\} $\implies$ & \{K.6.4\} &$0.1\%$ & $100 \%$ & $479.6$\\
	\{J.1, K.6.4\} $\implies$ & \{H.4.1\} & $0.1\%$ & $100 \%$ & $473.0$\\
	\{H.4.1, K.8.1\} $\implies$ & \{K.6.4\} &$ 0.2\%$ & $100 \%$ & $479.6$\\
	\{K.6.4, K.8.1\} $\implies$ & \{H.4.1\} &$ 0.2\%$ & $100 \%$ & $473.0$\\
	\{D.2.5,K.8.1\} $\implies$ & \{H.4.1\}  &$ 0.1\%$ & $100 \%$ & $473.0$\\
	\{D.2.5, H.4.1\} $\implies$ & \{K.8.1\} &$0.1\%$ & $100 \%$ & $567.6$\\
	\hline
\end{tabular}
 \caption{Assoziationsregeln der Unterthemen sortiert nach Konfidenz mit  Support: 0.1 \% und Konfidenz 50 \%}
\end{table}
\begin{table}[H]
\centering % used for centering table
\begin{tabular}{|rl|l|l|l|}
	\hline
	\textbf{Regel}& &\textbf{sup} &section\textbf{conf} &\textbf{lift}\\
	\hline
	\{J.1, K.6.4\} $\implies$ &\{D.2.9\} &0.1\% & 100 \% & 577.2 \\
	\{H.4.1, J.1, K.6.4\} $\implies$ & \{D.2.9\} &0.1\%& 100\% & 577.2\\
	\{D.2.5, H.4.1\} $\implies$ & \{K.8.1\} &0.1\%& 100\% & 567.6\\
	\{D.2.5, K.6.4\} $\implies$ & \{K.8.1\} &0.1\%& 100\% & 567.6\\
	\{D.2.5, H.4.1, K.6.4\} $\implies$ & \{K.8.1\} &0.1\%& 100 \% & 567.6\\
	\{H.4.1, J.1\} $\implies$ & \{D.2.9\} &0.1\%& 92.7 \% & 534.9\\
	\{D.2.9, H.4.1\} $\implies$ & \{J.1\} &0.1\%& 100 \% & 500.8\\
	\{D.2.9, H.4.1, K.6.4\} $\implies$ & \{J.1\} &0.1\%& 100\% & 500.8\\
	\{D.2.9, K.6.4\} $\implies$ & \{J.1\} &0.1\%& 97.4\% & 487.9\\
	\{D.2.9, J.1\} $\implies$ & \{K.6.4\} &0.1\%& 100\% & 479.6\\
	\hline
\end{tabular}
 \caption{Assoziationsregeln der Unterthemen sortiert nach Lift mit  Support: 0.1 \% und Konfidenz 50 \%}
\end{table}
\begin{table}[H]
\begin{tabular}{|rl|l|l|l|}
	\hline
	\textbf{Regel} & &\textbf{sup} &\textbf{conf} &\textbf{lift}\\
	\hline
	\small Math - Combinatoricscs $\implies$ & \small CS - Discrete Mathematics  &1.7\% &54 \% &7.9 \\
	\small CS - Social and Information Networks $\implies$ & \small Physics - Physicscs and Society &1.7\% &82 \% &25.5 \\
	\small Physics - Physics and Society $\implies$ & \small CS - Social and Information Networks &1.7\% &52 \% &25.5 \\
	\small F.4.1 $\implies$ & \small CS - Logic in CSomputer Science &1.6\% &78 \% &10.6 \\
	\small CS- Symbolicstems and Control $\implies$ & \small Math - Optimization and Control &1.5\% &88 \% &34 \\
	\small I.2.7 $\implies$ & \small CS - Computation and Languagesge &1.2\% &90 \% &16.9 \\
	\small Math - Optimization and Control $\implies$ & \small CS - Systems and Control  &1.5\% &56 \% &34 \\
	\small I.2.7(Natural Language Processing)  $\implies$ & \small CS - Computersmputation and Language  &1.2\% &91 \% &16.9 \\
	\small G.2.2(Graph Theory)   $\implies$ & \small CS - Discrete Mathematics  &0.8\% &59 \% &8.8 \\
	\small F.1.3(Complexity Measures)  $\implies$ & \small CS - Computational Complexity  &0.7\% &86 \% &12.6 \\
	\hline
\end{tabular}
 \caption{Assoziationsregeln der Unterthemen sortiert nach Support mit  Support: 0.1 \% und Konfidenz 50 \%}
\end{table}
\subsubsection{Mappings zwischen Klassifikationen}
\begin{table}[H]
\centering % used for centering table
\begin{tabular}{|rl|c|c|c|}
	\hline
	\textbf{Regel(ACM $\implies$ }  &\textbf{arXiv.org)} & {Support} &\textbf{Konfidenz} & \textbf{Lift}\\
	\hline
	\small G.4(Mathematical Software) $\implies$ & \small CS - Mathematical Softwareftware & 0.1\% &51\% &67.4  \\
	\small K.4.m(Computers AND Society) $\implies$ & \small CS - Computers and Society &0.2 \% &96 \%  &53.4 \\
	\small I.2.9(Robotics) $\implies$ & \small CS - Robotics &0.1 \% & 75 \%  &52.1 \\
	\small H.3.7(Digital Libraries) $\implies$ & \small CS - Digital Libraries &0.3 \% &89 \% &48.7 \\
	\small I.1.2(Symbolic Manipulation - Algorithm) $\implies$ & \small CS - Symbolic Computation  &0.1 \% &60 \%  &41.3 \\
	\small I.2.11(Distributed Artificial Intelligence) $\implies$ & \small CS - Multiagent Systems  &0.2 \% &53 \%  &35.9 \\
	\small H.5.2(User Interfaces) $\implies$ & \small CS - Human-Computer Interaction &0.2 \% &58 \%  &33.7 \\
	\small I.3.5(Computational Geometry) $\implies$ & \small CS - Computational Geometry &0.3 \% &88 \%  &31.2 \\
	\small H.2.3(Datebase Managment - Languages) $\implies$ & \small CS - Databases &0.2 \% &91 \%  &30.6 \\
	\hline
\end{tabular}
 \caption{Assoziationsregeln der Unterthemen Support: 1 \% und Konfidenz 50 \%}
\end{table}

\section{Further Research und m�gliche Anwendung}
bl� bl� bl� bo�

\section{Zusammenfassung}
Wie in dieser Ausarbeitung gezeigt, ist die Assoziationsanalyse der Themenwissenschaftlichen Publikationen eine wichtige Methode der bibliometrischen Untersuchungen.
Durch die Betrachtung der vergebenen Subjects f�r den arxiv-Datensatz haben wir bedeutende Erkenntnisse �ber die Verwandschaft manchen Themen als auch �ber die Entwicklung �ber die Zeit von bestimmten Themengebieten gewonnen.
Anlehnend and die Ergebnisse, die dieses Paper pr�sentiert, k�nnen standartisierten Mappings zwischen verschiedenen Klassifikationen, Themenvorschlag- sowie Paperrecommendersysteme entwickelt werden.
Um diese erfolgreich umzusetzen, bietet sich an, neben der Assoziationsanalyse von Themen auch weitere Methoden, wie z.B. Autorenanalyse oder Untersuchung der Schl�sselw�rter des Titels und/oder des Abstracts, zu verwenden.

%* was gewinnen wir durch Assoziationsanalyse von Themen
%* was haben wir gemacht
%* hinweis auf weiterf�hrende recherche

%In a general way,
%
%restate your topic and why it is important,
%restate your thesis/claim,
%address opposing viewpoints and explain why readers should align with your position,
%call for action or overview future research possibilities.
%
%from my specific topic to more general matters
%say what you have already said but do it quickly, sharply and in different words


\cite*{HasChe11}
\cite*{DBLP:journals/jmlr/HahslerCHB11}
\cite*{HahGruHorBuc11}
\newpage

\bibliographystyle{lni}
\bibliography{lnitemplate}

\end{document}
