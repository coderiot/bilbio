\begin{Verbatim}[commandchars=\\\{\}, fontsize=\footnotesize, frame=single]
\PY{n+nt}{<metadata}\PY{n+nt}{>}
 \PY{n+nt}{<oai\PYZus{}dc:dc} \PY{n+na}{xmlns:oai\PYZus{}dc=}\PY{l+s}{"http://www.openarchives.org/OAI/2.0/oai\PYZus{}dc/"} 
   \PY{n+na}{xmlns:dc=}\PY{l+s}{"http://purl.org/dc/elements/1.1/"} 
   \PY{n+na}{xsi:schemaLocation=}\PY{l+s}{"http://www.openarchives.org/OAI/2.0/oai\PYZus{}dc.xsd"}\PY{n+nt}{>}
 \PY{n+nt}{<dc:title}\PY{n+nt}{>}Sparsity-certifying Graph Decompositions\PY{n+nt}{</dc:title>}
 \PY{n+nt}{<dc:creator}\PY{n+nt}{>}Streinu, Ileana\PY{n+nt}{</dc:creator>}
 \PY{n+nt}{<dc:creator}\PY{n+nt}{>}Theran, Louis\PY{n+nt}{</dc:creator>}
 \PY{n+nt}{<dc:subject}\PY{n+nt}{>}Mathematics - Combinatorics\PY{n+nt}{</dc:subject>}
 \PY{n+nt}{<dc:subject}\PY{n+nt}{>}Computer Science - Computational Geometry\PY{n+nt}{</dc:subject>}
 \PY{n+nt}{<dc:subject}\PY{n+nt}{>}05C85\PY{n+nt}{</dc:subject>}
 \PY{n+nt}{<dc:subject}\PY{n+nt}{>}05C70\PY{n+nt}{</dc:subject>}
 \PY{n+nt}{<dc:subject}\PY{n+nt}{>}68R10\PY{n+nt}{</dc:subject>}
 \PY{n+nt}{<dc:subject}\PY{n+nt}{>}05B35\PY{n+nt}{</dc:subject>}
 \PY{n+nt}{<dc:description}\PY{n+nt}{>}  
    We describe a new algorithm, the \PYZdl{}(k,\PYZbs{}ell)\PYZdl{}-pebble game with colors, 
    and use it obtain a characterization of the family of 
    \PYZdl{}(k,\PYZbs{}ell)\PYZdl{}-sparse graphs and algorithmic solutions to a family
    of problems concerning tree decompositions of graphs. 
    Special instances of sparse graphs appear in rigidity theory and have 
    received increased attention in recent years. In particular, 
    our colored pebbles generalize and strengthen the previous results 
    of Lee and Streinu and give a new proof of the Tutte-Nash-Williams 
    characterization of arboricity. We also present a new decomposition 
    that certifies sparsity based on the \PYZdl{}(k,\PYZbs{}ell)\PYZdl{}-pebble 
    game with colors. Our work also exposes connections between pebble
    game algorithms and previous sparse graph algorithms by Gabow, Gabow
    and Westermann and Hendrickson.
 \PY{n+nt}{</dc:description>}
 \PY{n+nt}{<dc:description}\PY{n+nt}{>}
   Comment: To appear in Graphs and Combinatorics
 \PY{n+nt}{</dc:description>}
 \PY{n+nt}{<dc:date}\PY{n+nt}{>}2007-03-30\PY{n+nt}{</dc:date>}
 \PY{n+nt}{<dc:date}\PY{n+nt}{>}2008-12-13\PY{n+nt}{</dc:date>}
 \PY{n+nt}{<dc:type}\PY{n+nt}{>}text\PY{n+nt}{</dc:type>}
 \PY{n+nt}{<dc:identifier}\PY{n+nt}{>}http://arxiv.org/abs/0704.0002\PY{n+nt}{</dc:identifier>}
 \PY{n+nt}{</oai\PYZus{}dc:dc>}
\PY{n+nt}{</metadata>}
\end{Verbatim}
