\documentclass[12pt, xcolor=table]{beamer}
\usepackage{graphicx}
\usepackage[ngerman]{babel}
\usepackage[utf8]{inputenc}
\usepackage{amsmath}
\usepackage{amssymb}
\usepackage{listings}
\usepackage{hyperref}
\usepackage{fancyvrb}
\usepackage{color}
\usepackage{alltt}

\usepackage[percent]{overpic}
\usepackage[footnotesize, bf]{caption}
\input{theme.tex}
\input{syntax}
\renewcommand{\footnotesize}{\tiny}
\begin{document}
\title{Algorithmen und Analyse auf bibliographischen Daten}
\author{peterr und Lusy}
\date{\today}

\begin{frame}
	\titlepage
\end{frame}

\begin{frame}
	\frametitle{Eigenschaften des Datensatzes}
	\begin{itemize}
		\item  enthält ca. $706\,000$ Einträge
		\item  mit 19 verschiedenen Themengebieten
		\item  nur der Themenbereich Physik wird in Themengruppen unterteilt
		\item  11 Einträge ohne Informationen
		\item  Publikationen haben im Durchschnitt 1.3 und maximal 9 Themen
	\end{itemize}
\end{frame}

\begin{frame}
	\frametitle{Verteilung der Themen}
	\begin{center}
		\includegraphics[scale=0.35]{../../visual/treeParent2.png}
	\end{center}
\end{frame}
\begin{frame}[fragile]
	\frametitle{Aufbau des Datensatzes}
	\begin{block}{Header}
		\begin{Verbatim}[commandchars=\\\{\}, fontsize=\footnotesize, frame=single]
\PY{n+nt}{<header}\PY{n+nt}{>}
  \PY{n+nt}{<identifier}\PY{n+nt}{>}oai:arXiv.org:0704.0002\PY{n+nt}{</identifier>}
  \PY{n+nt}{<datestamp}\PY{n+nt}{>}2008-12-13\PY{n+nt}{</datestamp>}
  \PY{n+nt}{<setSpec}\PY{n+nt}{>}cs\PY{n+nt}{</setSpec>}
  \PY{n+nt}{<setSpec}\PY{n+nt}{>}math\PY{n+nt}{</setSpec>}
\PY{n+nt}{</header>}
\end{Verbatim}

	\end{block}
	\begin{block}{Metadaten}
		\begin{Verbatim}[commandchars=\\\{\}, fontsize=\tiny, frame=single]
\PY{n+nt}{<dc:title}\PY{n+nt}{>}Titel des Papers\PY{n+nt}{</dc:title>}
\PY{n+nt}{<dc:creator}\PY{n+nt}{>}Author 1\PY{n+nt}{</dc:creator>}
\PY{n+nt}{<dc:creator}\PY{n+nt}{>}Author 2\PY{n+nt}{</dc:creator>}
\PY{n+nt}{<dc:subject}\PY{n+nt}{>}\textcolor{red}{\bf{Physics - Optics}}\PY{n+nt}{</dc:subject>}
\PY{n+nt}{<dc:subject}\PY{n+nt}{>}\textcolor{red}{\bf{Mathematics - Combinatorics}}\PY{n+nt}{</dc:subject>}
\PY{n+nt}{<dc:description}\PY{n+nt}{>}Description\PY{n+nt}{</dc:description>}
\PY{n+nt}{<dc:description}\PY{n+nt}{>}Comment\PY{n+nt}{</dc:description>}
\PY{n+nt}{<dc:date}\PY{n+nt}{>}\textcolor{red}{\bf{2007-04-02}}\PY{n+nt}{</dc:date>}
\PY{n+nt}{<dc:date}\PY{n+nt}{>}\textcolor{red}{\bf{2007-07-24}}\PY{n+nt}{</dc:date>}
\PY{n+nt}{<dc:type}\PY{n+nt}{>}text\PY{n+nt}{</dc:type>}
\PY{n+nt}{<dc:identifier}\PY{n+nt}{>}http://arxiv.org/abs/0704.0001\PY{n+nt}{</dc:identifier>}
\PY{n+nt}{<dc:identifier}\PY{n+nt}{>}Phys.Rev.D76:013009,2007\PY{n+nt}{</dc:identifier>}
\end{Verbatim}

	\end{block}
\end{frame}
\begin{frame} [fragile]
	\frametitle{Eigenschaften des CS-Anteils}
	 {\small
     enthält ca. $34\,000$ Einträge\\
    }
    %\begin{block}
     \begin{alltt}\tiny
most frequent items:
                  Computer Science - Information Theory
                                                   6646
             Computer Science - Artificial Intelligence
                                                   3045
      Computer Science - Data Structures and Algorithms
                                                   2878
Computer Science - Networking and Internet Architecture
                                                   2660
           Computer Science - Logic in Computer Science
                                                   2498
                                                (Other)
                                                  58364

element (itemset/transaction) length distribution:
sizes
    1     2     3     4     5     6     7     8     9    10    11    12    13
13906  9209  5443  2802  1351   684   317   173    73    41    16    15     7
   14    15    16    17    19    21    22
    6     5     1     1     1     1     1

   Min. 1st Qu.  Median    Mean 3rd Qu.    Max.
  1.000   1.000   2.000   2.234   3.000  22.000

\end{alltt}

	%\end{block}
\end{frame}

\begin{frame}
	\frametitle{Verteilung der Themen in Computer Science}
	\begin{center}
		\includegraphics[scale=0.4]{../../visual/csFrequent_filter_acm_and_msc.png}
	\end{center}
\end{frame}
\begin{frame}
	\frametitle{Verteilung der Themen in Computer Science}
    VOSviewer demo
	\begin{center}
		\includegraphics[scale=0.25]{../../visual/cs_subs_cluster_density.png}
	\end{center}
\end{frame}
\begin{frame}
	\frametitle{Interpretation der Ergebnisse}
    \begin{center}
        \begin{description}
			\item [Support  ] \hfill \\
                relative Häufigkeit der Menge in den Daten\\
                $supp = \frac{\# gemeinsames Vorkommen Bestimmten Subjects}{\# alle Paper Im Datensatz}$\\
			\item [Konfidenz ] \hfill \\
                Häufigkeit des gemeinsamen Auftretens von X und Y, unter der Bedingung, dass X auftritt\\
                $conf(X\Rightarrow Y) = \frac{supp(X\cup Y)}{supp(X)} $\\

			\item [Lift ] \hfill \\
                Bedeutung der Regel\\
                $lift(X\Rightarrow Y) = \frac{supp(X\cup Y)}{supp(Y)\times supp(X)} $\\
	    \end{description}
    \end{center}
\end{frame}
\begin{frame}[fragile]
    \frametitle{Interpretation der Ergebnisse}
    \begin{block}{Arules für supp=0.01}
    	\rowcolors[]{1}{blue!20}{blue!10}
	\begin{tabular}{rccc}
		\tiny \textbf{Regel} &\tiny \textbf{sup} &\tiny \textbf{conf} &\tiny \textbf{lift}\\
		\hline
		\tiny I.2.7 $\implies$ CS - Computation and Language &\tiny 1.2\% &\tiny 90 \% &\tiny 16.9 \\
		\tiny CS- Systems and Control $\implies$ Math - Optimization and Control &\tiny 1.5\% &\tiny 88 \% &\tiny 34 \\
		\tiny Math - Optimization and Control $\implies$ CS - Systems and Control  &\tiny 1.5\% &\tiny 56 \% &\tiny 34 \\
		\tiny CS - Social and Information Networks $\implies$ Physics - Physics and Society &\tiny 1.7\% &\tiny 82 \% &\tiny 25.5 \\
		\tiny Physics - Physics and Society $\implies$ CS - Social and Information Networks &\tiny 1.7\% &\tiny 52 \% &\tiny 25.5 \\
		\tiny F.4.1 $\implies$ CS - Logic in Computer Science &\tiny 1.6\% &\tiny 78 \% &\tiny 10.6 \\
		\tiny Math - Combinatorics $\implies$ CS - Discrete Mathematics  &\tiny 1.7\% &\tiny 54 \% &\tiny 7.9 \\
	\end{tabular}

	\end{block}
    {\footnotesize
        F. - Theory of Computation\\
        F.4. - Mathematical Logic and Formal Languages\\
        F.4.1. - Mathematical Logic\\
        I. - Computing Methodologies\\
        I.2. - Artificial Intelligence\\
        I.2.7. - Natural Language Processing\\
    }
\end{frame}
\begin{frame}
	\frametitle{Arules für supp=0.01}
	\begin{center}
		\includegraphics[scale=0.5]{../../visual/plot_matrix_grouped_7rules.png}
	\end{center}
\end{frame}
\begin{frame}
    \frametitle{Arules für supp=0.001, Top 10 Lift}
    \begin{alltt}\tiny
lhs                                  rhs              support      confidence    lift

1  {J.1, K.6.4}                   => {D.2.9}          0.001115908  1.0000000  577.169492
{ADMINISTRATIVE DATA PROCESSING, System Management => Management}

2  {H.4.1, J.1, K.6.4}            => {D.2.9}          0.001115908  1.0000000  577.169492
{Office Automation, ADMINISTRATIVE DATA PROCESSING, System Management => Management}

3  {D.2.5, H.4.1}                 => {K.8.1}          0.001174640  1.0000000  567.550000
{Testing and Debugging, Office Automation => Application Packages}

4  {D.2.5, K.6.4}                 => {K.8.1}          0.001174640  1.0000000  567.550000
{Testing and Debugging, System Management => Application Packages}

5  {D.2.5, H.4.1, K.6.4}          => {K.8.1}          0.001174640  1.0000000  567.550000
{Testing and Debugging, Office Automation, System Management => Application Packages}

6  {H.4.1, J.1}                   => {D.2.9}          0.001115908  0.9268293  534.937578
{Office Automation, ADMINISTRATIVE DATA PROCESSING => Management}

7  {D.2.9, H.4.1}                 => {J.1}            0.001115908  1.0000000  500.779412
{Management, Office Automation => ADMINISTRATIVE DATA PROCESSING}

8  {D.2.9, H.4.1, K.6.4}          => {J.1}            0.001115908  1.0000000  500.779412
{Management, Office Automation, System Management => ADMINISTRATIVE DATA PROCESSING}

9  {D.2.9, K.6.4}                 => {J.1}            0.001115908  0.9743590  487.938914
{Management, System Management => ADMINISTRATIVE DATA PROCESSING}

10 {D.2.9, J.1}                   => {K.6.4}          0.001115908  1.0000000  479.619718
{Management, ADMINISTRATIVE DATA PROCESSING => System Management}
\end{alltt}

\end{frame}
\begin{frame}
    \frametitle{Arules für supp=0.001, Top 10 Support}
    \begin{alltt}\tiny
lhs                                                   rhs                                                      support      confidence       lift
1  {Mathematics-Combinatorics}                         => {Computer Science - Discrete Mathematics}             0.017061639  0.5354839   7.924742
2  {Computer Science-Social and Information Networks}  => {Physics - Physics and Society}                       0.017002907  0.8212766  25.540577
3  {Physics-Physics and Society}                       => {Computer Science - Social and Information Networks}  0.017002907  0.5287671  25.540577
4  {F.4.1} (Mathematical Logic)                        => {Computer Science - Logic in Computer Science}        0.016033830  0.7777778  10.602749
5  {Computer Science-Systems and Control}              => {Mathematics - Optimization and Control}              0.014506798  0.8758865  34.126503
6  {Mathematics-Optimization and Control}              => {Computer Science - Systems and Control}              0.014506798  0.5652174  34.126503
7  {I.2.7} (Natural Language Processing)               => {Computer Science - Computation and Language}         0.012921035  0.9147609  16.929540
8  {G.2.2} (Graph Theory)                              => {Computer Science - Discrete Mathematics}             0.007928817  0.5934066   8.781953
9  {I.2.4}                                             => {Computer Science - Artificial Intelligence}          0.007077203  0.8169492   9.136148
(Knowledge Representation Formalism and Methods)
10 {F.1.3} (Complexity Measures and Classes)           => {Computer Science - Computational Complexity}         0.006812909  0.8560886  12.565683
\end{alltt}

\end{frame}
\begin{frame}
    \frametitle{Arules für supp=0.001, Top 10 Confidence}
    \begin{alltt}\tiny
lhs                           rhs                support      confidence    lift

1   {D.2.9, J.1}           => {H.4.1}            0.001115908  1.0000000  472.958333
{Management, ADMINISTRATIVE DATA PROCESSING => Office Automation}

2   {D.2.9, H.4.1}         => {J.1}              0.001115908  1.0000000  500.779412
{Management, Office Automation => ADMINISTRATIVE DATA PROCESSING}

3   {D.2.9, J.1}           => {K.6.4}            0.001115908  1.0000000  479.619718
{Management, ADMINISTRATIVE DATA PROCESSING => System Management}

4   {J.1, K.6.4}           => {D.2.9}            0.001115908  1.0000000  577.169492
{ADMINISTRATIVE DATA PROCESSING, System Management => Management}

5   {D.2.9, H.4.1}         => {K.6.4}            0.001115908  1.0000000  479.619718
{Management, Office Automation => System Management}

6   {J.1, K.6.4}           => {H.4.1}            0.001115908  1.0000000  472.958333
{ADMINISTRATIVE DATA PROCESSING, System Management => Office Automation}

7   {H.4.1, K.8.1}         => {K.6.4}            0.001615129  1.0000000  479.619718
{Office Automation, Application Packages => System Management}

8   {K.6.4, K.8.1}         => {H.4.1}            0.001615129  1.0000000  472.958333
{System Management, Application Packages => Office Automation}

9   {D.2.5,K.8.1}          => {H.4.1}            0.001174640  1.0000000  472.958333
{Testing and Debugging, Application Packages => Office Automation}

10  {D.2.5, H.4.1}         => {K.8.1}            0.001174640  1.0000000  567.550000
{Testing and Debugging, Office Automation => Application Packages}
\end{alltt}

\end{frame}
\begin{frame}
    \frametitle{Arules für supp=0.001}
    \begin{center}
		\includegraphics[scale=0.5]{../../visual/graph_218rules.png}
	\end{center}
\end{frame}
\begin{frame}
	\frametitle{Entwicklung über die Zeit 1}
	\begin{center}
		\includegraphics[scale=0.4]{../../visual/trend/csmath.png}
	\end{center}
\end{frame}
\begin{frame}
	\frametitle{Entwicklung über die Zeit 2}
	\begin{center}
		\includegraphics[scale=0.4]{../../visual/trend/csph.png}
	\end{center}
\end{frame}
\begin{frame}
	\frametitle{Entwicklung über die Zeit 3}
	\begin{center}
		\includegraphics[scale=0.4]{../../visual/trend/combcs.png}
	\end{center}
\end{frame}


\begin{frame}
	\frametitle{Mappings zwischen Klassifikationen}
	Top 10(insg. ca. 60)
	\rowcolors[]{1}{blue!20}{blue!10}
	\begin{tabular}{lccc}
		\tiny\textbf{Regel(ACM $\implies$ Arxiv.org)} &\tiny \textbf{Support} &\tiny \textbf{Konfidenz} & \tiny \textbf{Lift}\\
		\hline
		\tiny G.4(Mathematical Software) $\implies$ CS - Mathematical Software & \tiny 0.1\% &\tiny 51\% &\tiny 67.4  \\
		\tiny K.4.m(COMPUTERS AND SOCIETY) $\implies$ CS - Computers and Society &\tiny 0.2 \% &\tiny 96 \%  &\tiny 53.4 \\
		\tiny I.2.9(Robotics) $\implies$ CS - Robotics &\tiny 0.1 \% &\tiny 75 \%  &\tiny 52.1 \\
		\tiny H.3.7(Digital Libraries) $\implies$ CS - Digital Libraries &\tiny 0.3 \% &\tiny 89 \% &\tiny 48.7 \\
		\tiny I.1.2(Symbolic Manipulation - Algorithm) $\implies$ CS - Symbolic Computation  &\tiny 0.1 \% &\tiny 60 \%  &\tiny 41.3 \\
		\tiny I.2.11(Distributed Artificial Intelligence) $\implies$ CS - Multiagent Systems  &\tiny 0.2 \% &\tiny 53 \%  &\tiny 35.9 \\
		\tiny H.5.2(User Interfaces) $\implies$ CS - Human-Computer Interaction &\tiny 0.2 \% &\tiny 58 \%  &\tiny 33.7 \\
		\tiny I.3.5(Computational Geometry) $\implies$ CS - Computational Geometry &\tiny 0.3 \% &\tiny 88 \%  &\tiny 31.2 \\
		\tiny H.2.3(Datebase Managment - Languages) $\implies$ CS - Databases &\tiny 0.2 \% &\tiny 91 \%  &\tiny 30.6 \\
	\end{tabular}
\end{frame}
\begin{frame}
	\frametitle{Einschränkungen für Mappings}
    \begin{itemize}
	\item nur ein Subject auf jeder Seite
	\item Mappings nur von ACM nach Arxiv möglich
	\item Konflikte, durch Mappings von gleichen ACM-Klassen
	\item ab einen Schwellwert für den Lift werden die Mappings zu ungenau
    \end{itemize}
\end{frame}
\begin{frame}
	\frametitle{Anwendung für Mappings}
    \begin{itemize}
	\item automatische Anpassung von ACM-Klassifikation
	\item Reduzierung von Subjects pro Paper für Analyse
	\item recommendation
    \end{itemize}
\end{frame}
\begin{frame}[fragile]
    \frametitle{Nutzen der verschiedenen Klassifikationen}
     Welche Regeln bleiben übrig, wenn ohne ACM Subjects?\\
    \begin{block}{Arules für supp=0.001}
    	\begin{alltt}\tiny

lhs                                                            rhs                                                      support confidence      lift
1  {Mathematics - Dynamical Systems, Mathematics - Optimization and Control}                    => {Computer Science - Systems and Control}             0.001175468  0.8333333 50.279255
2  {Computer Science - Systems and Control, Mathematics - Probability}                                 => {Mathematics - Optimization and Control}             0.001057921  0.9729730 37.882491
3  {Computer Science - Computational Complexity, Condensed Matter - Disordered Systems and Neural Networks} => {Condensed Matter - Statistical Mechanics}           0.001351788  0.6301370 37.487642
4  {Mathematics - Commutative Algebra}                         => {Computer Science - Symbolic Computation}            0.001322401  0.5555556 37.287968
5  {Computer Science - Systems and Control, Mathematics - Dynamical Systems}                           => {Mathematics - Optimization and Control}             0.001175468  0.9523810 37.080745
6  {Mathematics - Optimization and Control, Mathematics - Probability}                                 => {Computer Science - Systems and Control}             0.001057921  0.5714286 34.477204
7  {Computer Science - Systems and Control}                    => {Mathematics - Optimization and Control}             0.014517030  0.8758865 34.102451
8  {Mathematics - Optimization and Control}                    => {Computer Science - Systems and Control}             0.014517030  0.5652174 34.102451
9  {Mathematics - Numerical Analysis}                          => {Computer Science - Numerical Analysis}              0.004672485  0.5317726 33.887058
10 {Computer Science - Information Theory, Computer Science - Systems and Control}                    => {Mathematics - Optimization and Control}             0.001057921  0.8000000 31.147826
\end{alltt}

	\end{block}
\end{frame}
\end{document}
