\documentclass[12pt, xcolor=table]{beamer}
\usepackage{graphicx}
\usepackage[ngerman]{babel}
\usepackage[utf8]{inputenc}
\usepackage{amsmath}
\usepackage{amssymb}
\usepackage{listings}
\usepackage{hyperref}
\usepackage{fancyvrb}
\usepackage{color}
\usepackage{alltt}

\usepackage[percent]{overpic}
\usepackage[footnotesize, bf]{caption}
\input{theme.tex}
\input{syntax}
\renewcommand{\footnotesize}{\tiny}
\begin{document}
\title{Algorithmen und Analyse auf bibliographischen Daten}
\author{peterr und Lusy}
\date{\today}

\begin{frame}
	\titlepage
\end{frame}

\begin{frame}
	\frametitle{Eigenschaften des Datensatzes}
	\begin{itemize}
		\item  enthält ca. $706\,000$ Einträge
		\item  mit 19 verschiedenen Themengebieten
		\item  nur der Themenbereich Physik wird in Themengruppen unterteilt
		\item  11 Einträge ohne Informationen
		\item  Publikationen haben im Durchschnitt 1.3 und maximal 9 Themen
	\end{itemize}
\end{frame}

\begin{frame}
	\frametitle{Verteilung der Themen}
	\begin{center}
		\includegraphics[scale=0.35]{../../visual/treeParent2.png}
	\end{center}
\end{frame}
\begin{frame}[fragile]
	\frametitle{Aufbau des Datensatzes}
	\begin{block}{Header}
		%\begin{Verbatim}[commandchars=\\\{\}, fontsize=\footnotesize, frame=single]
\PY{n+nt}{<header}\PY{n+nt}{>}
  \PY{n+nt}{<identifier}\PY{n+nt}{>}oai:arXiv.org:0704.0002\PY{n+nt}{</identifier>}
  \PY{n+nt}{<datestamp}\PY{n+nt}{>}2008-12-13\PY{n+nt}{</datestamp>}
  \PY{n+nt}{<setSpec}\PY{n+nt}{>}cs\PY{n+nt}{</setSpec>}
  \PY{n+nt}{<setSpec}\PY{n+nt}{>}math\PY{n+nt}{</setSpec>}
\PY{n+nt}{</header>}
\end{Verbatim}

	\end{block}
	\begin{block}{Metadaten}
		%\begin{Verbatim}[commandchars=\\\{\}, fontsize=\tiny, frame=single]
\PY{n+nt}{<dc:title}\PY{n+nt}{>}Titel des Papers\PY{n+nt}{</dc:title>}
\PY{n+nt}{<dc:creator}\PY{n+nt}{>}Author 1\PY{n+nt}{</dc:creator>}
\PY{n+nt}{<dc:creator}\PY{n+nt}{>}Author 2\PY{n+nt}{</dc:creator>}
\PY{n+nt}{<dc:subject}\PY{n+nt}{>}\textcolor{red}{\bf{Physics - Optics}}\PY{n+nt}{</dc:subject>}
\PY{n+nt}{<dc:subject}\PY{n+nt}{>}\textcolor{red}{\bf{Mathematics - Combinatorics}}\PY{n+nt}{</dc:subject>}
\PY{n+nt}{<dc:description}\PY{n+nt}{>}Description\PY{n+nt}{</dc:description>}
\PY{n+nt}{<dc:description}\PY{n+nt}{>}Comment\PY{n+nt}{</dc:description>}
\PY{n+nt}{<dc:date}\PY{n+nt}{>}\textcolor{red}{\bf{2007-04-02}}\PY{n+nt}{</dc:date>}
\PY{n+nt}{<dc:date}\PY{n+nt}{>}\textcolor{red}{\bf{2007-07-24}}\PY{n+nt}{</dc:date>}
\PY{n+nt}{<dc:type}\PY{n+nt}{>}text\PY{n+nt}{</dc:type>}
\PY{n+nt}{<dc:identifier}\PY{n+nt}{>}http://arxiv.org/abs/0704.0001\PY{n+nt}{</dc:identifier>}
\PY{n+nt}{<dc:identifier}\PY{n+nt}{>}Phys.Rev.D76:013009,2007\PY{n+nt}{</dc:identifier>}
\end{Verbatim}

	\end{block}
\end{frame}

\begin{frame}[fragile]
    \frametitle{Verteilung der Themen in Computer Science}
    \begin{block}{Subjects in CS: Summary}
    	\begin{alltt}\tiny
most frequent items:
                  Computer Science - Information Theory
                                                   6646
             Computer Science - Artificial Intelligence
                                                   3045
      Computer Science - Data Structures and Algorithms
                                                   2878
Computer Science - Networking and Internet Architecture
                                                   2660
           Computer Science - Logic in Computer Science
                                                   2498
                                                (Other)
                                                  58364

element (itemset/transaction) length distribution:
sizes
    1     2     3     4     5     6     7     8     9    10    11    12    13
13906  9209  5443  2802  1351   684   317   173    73    41    16    15     7
   14    15    16    17    19    21    22
    6     5     1     1     1     1     1

   Min. 1st Qu.  Median    Mean 3rd Qu.    Max.
  1.000   1.000   2.000   2.234   3.000  22.000

\end{alltt}

	\end{block}
\end{frame}

\begin{frame}
	\frametitle{Verteilung der Themen in Computer Science}
	\begin{center}
		\includegraphics[scale=0.4]{../../visual/csFrequent_filter_acm_and_msc.png}
	\end{center}
\end{frame}
\begin{frame}
	\frametitle{Verteilung der Themen in Computer Science}
    VOSviewer demo
	\begin{center}
		\includegraphics[scale=0.25]{../../visual/cs_subs_cluster_density.png}
	\end{center}
\end{frame}
\begin{frame}
	\frametitle{Entwicklung über die Zeit 1}
	\begin{center}
		\includegraphics[scale=0.4]{../../visual/trend/csmath.png}
	\end{center}
\end{frame}
\begin{frame}
	\frametitle{Entwicklung über die Zeit 2}
	\begin{center}
		\includegraphics[scale=0.4]{../../visual/trend/csph.png}
	\end{center}
\end{frame}
\begin{frame}
	\frametitle{Entwicklung über die Zeit 3}
	\begin{center}
		\includegraphics[scale=0.4]{../../visual/trend/combcs.png}
	\end{center}
\end{frame}
\begin{frame}
	\frametitle{Interpretation der Ergebnisse}
    \begin{block}{Assoziationsregeln - Kenngrößen}
    \begin{center}
        \begin{description}
			\item [Support  ] \hfill \\
                relative Häufigkeit der Menge in den Daten\\
                $supp = \frac{\# gemeinsames Vorkommen Bestimmten Subjects}{\# alle Paper Im Datensatz}$\\
			\item [Konfidenz ] \hfill \\
                Häufigkeit des gemeinsamen Auftretens von X und Y, unter der Bedingung, dass X auftritt\\
                $conf(X\Rightarrow Y) = \frac{supp(X\cup Y)}{supp(X)} $\\

			\item [Lift ] \hfill \\
                Bedeutung der Regel\\
                $lift(X\Rightarrow Y) = \frac{supp(X\cup Y)}{supp(Y)\times supp(X)} $\\
	    \end{description}
    \end{center}
    \end{block}
\end{frame}
\begin{frame}[fragile]
    \frametitle{Interpretation der Ergebnisse}
    \begin{block}{Arules für supp=0.01}
    	\rowcolors[]{1}{blue!20}{blue!10}
	\begin{tabular}{rccc}
		\tiny \textbf{Regel} &\tiny \textbf{sup} &\tiny \textbf{conf} &\tiny \textbf{lift}\\
		\hline
		\tiny I.2.7 $\implies$ CS - Computation and Language &\tiny 1.2\% &\tiny 90 \% &\tiny 16.9 \\
		\tiny CS- Systems and Control $\implies$ Math - Optimization and Control &\tiny 1.5\% &\tiny 88 \% &\tiny 34 \\
		\tiny Math - Optimization and Control $\implies$ CS - Systems and Control  &\tiny 1.5\% &\tiny 56 \% &\tiny 34 \\
		\tiny CS - Social and Information Networks $\implies$ Physics - Physics and Society &\tiny 1.7\% &\tiny 82 \% &\tiny 25.5 \\
		\tiny Physics - Physics and Society $\implies$ CS - Social and Information Networks &\tiny 1.7\% &\tiny 52 \% &\tiny 25.5 \\
		\tiny F.4.1 $\implies$ CS - Logic in Computer Science &\tiny 1.6\% &\tiny 78 \% &\tiny 10.6 \\
		\tiny Math - Combinatorics $\implies$ CS - Discrete Mathematics  &\tiny 1.7\% &\tiny 54 \% &\tiny 7.9 \\
	\end{tabular}

	\end{block}
    {\footnotesize
        F. - Theory of Computation\\
        F.4. - Mathematical Logic and Formal Languages\\
        F.4.1. - Mathematical Logic\\
        I. - Computing Methodologies\\
        I.2. - Artificial Intelligence\\
        I.2.7. - Natural Language Processing\\
    }
    %\begin{itemize}
            %\item Show rules - results from R
            %\item Visualize - choose an apropriate diagramm
    %\end{itemize}
\end{frame}
\begin{frame}
	\frametitle{Arules für supp=0.01}
	\begin{center}
		\includegraphics[scale=0.5]{../../visual/plot_matrix_grouped_7rules.png}
	\end{center}
\end{frame}
\begin{frame}
    \frametitle{Interpretation der Ergebnisse}
    \begin{block}{Arules für supp=0.001}
        \begin{itemize}
           \item Same as above
           \item Arules sorted according to supp, conf, lift
        \end{itemize}
    \end{block}
\end{frame}

\begin{frame}
	\frametitle{Umgang mit den Klassifikationen}
\end{frame}
\begin{frame}
	\frametitle{Nutzen der verschiedenen Klassifikationen}
    Mappings zwischen den Klassifikationen\\
    Nur möglich, wenn auf jeder Seite 1 Subject steht\\
    mappings sind in results zu finden\\
    Welche Regeln bleiben übrig, wenn ohne ACM Subjects?
\end{frame}
%\begin{frame}
	%\frametitle{Eigenschaften des Datensatzes}
	%\begin{itemize}
		%\item  enthält ca. $706\,000$ Einträge 
		%\item  mit 19 verschiedenen Themengebieten 
		%\item  nur der Themenbereich Physik wird in Themengruppen unterteilt
		%\item  11 Einträge ohne Informationen
		%\item  Publikationen haben im Durchschnitt 1.3 und maximal 9 Themen
	%\end{itemize} 
%\end{frame}
%\begin{frame}[fragile]
	%\frametitle{Aufbau des Datensatzes}
	%\begin{block}{Header}
		%\begin{Verbatim}[commandchars=\\\{\}, fontsize=\footnotesize, frame=single]
\PY{n+nt}{<header}\PY{n+nt}{>}
  \PY{n+nt}{<identifier}\PY{n+nt}{>}oai:arXiv.org:0704.0002\PY{n+nt}{</identifier>}
  \PY{n+nt}{<datestamp}\PY{n+nt}{>}2008-12-13\PY{n+nt}{</datestamp>}
  \PY{n+nt}{<setSpec}\PY{n+nt}{>}cs\PY{n+nt}{</setSpec>}
  \PY{n+nt}{<setSpec}\PY{n+nt}{>}math\PY{n+nt}{</setSpec>}
\PY{n+nt}{</header>}
\end{Verbatim}

	%\end{block}
	%\begin{block}{Metadaten}
		%\begin{Verbatim}[commandchars=\\\{\}, fontsize=\tiny, frame=single]
\PY{n+nt}{<dc:title}\PY{n+nt}{>}Titel des Papers\PY{n+nt}{</dc:title>}
\PY{n+nt}{<dc:creator}\PY{n+nt}{>}Author 1\PY{n+nt}{</dc:creator>}
\PY{n+nt}{<dc:creator}\PY{n+nt}{>}Author 2\PY{n+nt}{</dc:creator>}
\PY{n+nt}{<dc:subject}\PY{n+nt}{>}\textcolor{red}{\bf{Physics - Optics}}\PY{n+nt}{</dc:subject>}
\PY{n+nt}{<dc:subject}\PY{n+nt}{>}\textcolor{red}{\bf{Mathematics - Combinatorics}}\PY{n+nt}{</dc:subject>}
\PY{n+nt}{<dc:description}\PY{n+nt}{>}Description\PY{n+nt}{</dc:description>}
\PY{n+nt}{<dc:description}\PY{n+nt}{>}Comment\PY{n+nt}{</dc:description>}
\PY{n+nt}{<dc:date}\PY{n+nt}{>}\textcolor{red}{\bf{2007-04-02}}\PY{n+nt}{</dc:date>}
\PY{n+nt}{<dc:date}\PY{n+nt}{>}\textcolor{red}{\bf{2007-07-24}}\PY{n+nt}{</dc:date>}
\PY{n+nt}{<dc:type}\PY{n+nt}{>}text\PY{n+nt}{</dc:type>}
\PY{n+nt}{<dc:identifier}\PY{n+nt}{>}http://arxiv.org/abs/0704.0001\PY{n+nt}{</dc:identifier>}
\PY{n+nt}{<dc:identifier}\PY{n+nt}{>}Phys.Rev.D76:013009,2007\PY{n+nt}{</dc:identifier>}
\end{Verbatim}

	%\end{block}
%\end{frame}
%\begin{frame}
	%\frametitle{Parsen der Daten}
	%\begin{itemize}
		%\item  Parser in Python geschrieben 
		%\item  kompletter Datensatz in den Speicher
			%\begin{itemize}
				%\item Overhead des XML-Parser nicht beachtet
			%\end{itemize}
		%\item iterativer Ansatz \footnote{http://www.ibm.com/developerworks/xml/library/x-hiperfparse/}
		%\item benötigt ca. 70 Sekunden für 1.2 GB
	%\end{itemize}
%\end{frame}
%\begin{frame}
	%\frametitle{Verteilung der Themen}
	%\begin{center}
		%\includegraphics[scale=0.35]{../../visual/treeParent.png}
	%\end{center}
%\end{frame}
%\begin{frame}
	%\frametitle{Aufschlüsselung von physics}
	%\begin{center}
		%\includegraphics[scale=0.45]{../../visual/setSpecFreq.png}
	%\end{center}
%\end{frame}
%\begin{frame}
	%\frametitle{Häufigkeit von Themen pro Publikation}
	%\begin{columns}
		%\column{.5\textwidth}
    %\includegraphics[scale=0.35]{../../visual/piechart.png}
		%\column{.5\textwidth}
    %\includegraphics[scale=0.25]{../../visual/pieSubplot.png}
	%\end{columns}
%\end{frame}
%\begin{frame}
	%\frametitle{Was sind Assoziationsregeln?}
	%\begin{itemize}
		%\item bestimmen Korrelation des Auftretes von Mengen
		%\item Regel der Form "Wenn Menge A, dann Menge B"
		%\item Kenngrößen
		%\begin{itemize}
			%\item Support - relative Häufigkeit der Menge in den Daten
			 %\item Konfidenz - Häufikeit des gemeinsames Auftretens von A und B, unter der Bedingung das A auftritt
			 %\item Lift - Bedeutung der Regel
	%\end{itemize}
	%\end{itemize}
%\end{frame}
%\begin{frame}
	%\frametitle{Assoziationsregeln - aller Themen}
	%\begin{center}
	%\begin{table}
	%\rowcolors[]{1}{blue!20}{blue!10}
	%\begin{tabular}{rccc}
		%\tiny\textbf{Regel} &\tiny \textbf{Support} &\tiny \textbf{Konfidenz} & \tiny \textbf{Lift}\\
		%\hline
		%\tiny math $\implies$ stat & \tiny 0.6\% &\tiny 64\% &\tiny 3.0  \\
		%\tiny physics:math-ph $\implies$ math &\tiny 3.8 \% &\tiny 100\% &\tiny 4.7 \\
		%\tiny physics:hep-th, physics:math-ph  $\implies$ math &\tiny 0.9 \% &\tiny 100\% &\tiny 4.7 \\
		%\tiny math, physics:hep-th  $\implies$ physics:math-ph  &\tiny 0.9 \% &\tiny 63\% &\tiny 16.3 \\
		%\tiny physics:gr-qc, physics:hep-th $\implies$ physics:hep-th &\tiny 0.6 \% &\tiny 72 \% &\tiny 6.1 \\
		%\tiny physics:gr-qc, physics:hep-th $\implies$ physics:astro-ph &\tiny 0.6 \% &\tiny 70 \% &\tiny 3.5 \\
		%\tiny physics:gr-qc, physics:astro-ph $\implies$ physics:hep-th &\tiny 0.9 \% &\tiny 50 \%  &\tiny 4.3 \\
		%\tiny physics:astro-ph, physics:hep-th $\implies$ physics:gr-qc  &\tiny 0.9 \% &\tiny 74 \% &\tiny 12.4 \\
	%\end{tabular}
	 %\caption*{Support: 0.5 \% und Konfidenz 50 \%}
	%\end{table}
	%\end{center}
%\end{frame}
%\begin{frame}
	%\frametitle{Assoziationsregeln - Oberthemen}
	%\begin{center}
	%\begin{table}
	%\rowcolors[]{1}{blue!20}{blue!10}
	%\begin{tabular}{rccc}
		%\tiny\textbf{Regel} &\tiny \textbf{Support} &\tiny \textbf{Konfidenz} & \tiny \textbf{Lift}\\
		%\hline
		%\tiny  $\emptyset \implies$ physics & \tiny 78\% &\tiny 78\% &\tiny 1.0  \\
		%\tiny stat $\implies$ math  &\tiny 0.6 \% &\tiny 63 \% &\tiny 3.0 \\
		%\tiny nlin $\implies$ physics  &\tiny 1.3 \% &\tiny 50 \% &\tiny 0.64 \\
		%\tiny math, nlin $\implies$ physics  &\tiny 0.4 \% &\tiny 83 \% &\tiny 1.1 \\
	%\end{tabular}
	 %\caption*{Support: 0.1 \% und Konfidenz 50 \%}
	%\end{table}
	%\end{center}
%\end{frame}
%\begin{frame}
	%\frametitle{Probleme}
	%\begin{itemize}
		%\item mehrere Datumsangaben
		%\item Themen in Metadaten nicht eindeutig
		%\begin{itemize}
			%\item unterschiedliche Kategorisierungen
			%\item auch in einem Eintrag
		%\end{itemize}
		%\item Themenbereiche nachzuschlagen ist aufwendig
	%\end{itemize}
%\end{frame}
%\begin{frame}
	%\frametitle{Weitere Analysen}
	%\begin{itemize}
		%\item Aufschlüsselung der Themenbereiche
		%\item Regeln für die Unterthemen
		%\item Algorithmus implementieren?
			%\begin{itemize}
				%\item AIS-Algorithmnus
				%\item Apriori-Algorithmus 
				%\item FPGrowth 
			%\end{itemize}
		%\item Entwicklung in Abhängigkeit von der Zeit
	%\end{itemize}
%\end{frame}
\end{document}
