\documentclass[12pt, xcolor=table]{beamer}
\usepackage{graphicx}
\usepackage[ngerman]{babel}
\usepackage[utf8]{inputenc}
\usepackage{amsmath}
\usepackage{amssymb}
\usepackage{listings}
\usepackage{hyperref}
\usepackage{fancyvrb}
\usepackage{color}

\usepackage[percent]{overpic}
\usepackage[footnotesize, bf]{caption}
\input{theme.tex}
\input{syntax}
\renewcommand{\footnotesize}{\tiny}
\begin{document}
\title{Algorithmen und Analyse auf bibliographischen Daten}   
\author{peterr und Lusy} 
\date{\today} 

\begin{frame}
	\titlepage
\end{frame}

\begin{frame}
	\frametitle{Eigenschaften des Datensatzes}
	\begin{itemize}
		\item  enthält ca. $706\,000$ Einträge 
		\item  mit 19 verschiedenen Themengebieten 
		\item  nur der Themenbereich Physik wird in Themengruppen unterteiltt
		\item  11 Einträge ohne Informationen
		\item  die Publikationen haben im Durchschnitt 1.3 und maximal 9 Themen
	\end{itemize} 
\end{frame}
\begin{frame}[fragile]
	\frametitle{Aufbau der Datensatz}
	\begin{block}{Header}
	\begin{Verbatim}[commandchars=\\\{\}, fontsize=\tiny, frame=single]
		\PY{n+nt}{<identifier}\PY{n+nt}{>}oai:arXiv.org:0704.0001\PY{n+nt}{</identifier>}
		\PY{n+nt}{<datestamp}\PY{n+nt}{>}2007-07-24\PY{n+nt}{</datestamp>}
		\PY{n+nt}{<setSpec}\PY{n+nt}{>}\textcolor{red}{\bf{physics:physics}}\PY{n+nt}{</setSpec>}
		\PY{n+nt}{<setSpec}\PY{n+nt}{>}\textcolor{red}{\bf{math}}\PY{n+nt}{</setSpec>}
	\end{Verbatim}
	\end{block}
	\begin{block}{Metadaten}
		\begin{Verbatim}[commandchars=\\\{\}, fontsize=\tiny, frame=single]
			\PY{n+nt}{<dc:title}\PY{n+nt}{>}Titel des Papers\PY{n+nt}{</dc:title>}
			\PY{n+nt}{<dc:creator}\PY{n+nt}{>}Author 1\PY{n+nt}{</dc:creator>}
			\PY{n+nt}{<dc:creator}\PY{n+nt}{>}Author 2\PY{n+nt}{</dc:creator>}
			\PY{n+nt}{<dc:subject}\PY{n+nt}{>}\textcolor{red}{\bf{Physics - Optics}}\PY{n+nt}{</dc:subject>}
			\PY{n+nt}{<dc:subject}\PY{n+nt}{>}\textcolor{red}{\bf{Mathematics - Combinatorics}}\PY{n+nt}{</dc:subject>}
			\PY{n+nt}{<dc:description}\PY{n+nt}{>}Description\PY{n+nt}{</dc:description>}
			\PY{n+nt}{<dc:description}\PY{n+nt}{>}Comment\PY{n+nt}{</dc:description>}
			\PY{n+nt}{<dc:date}\PY{n+nt}{>}\textcolor{red}{\bf{2007-04-02}}\PY{n+nt}{</dc:date>}
			\PY{n+nt}{<dc:date}\PY{n+nt}{>}\textcolor{red}{\bf{2007-07-24}}\PY{n+nt}{</dc:date>}
			\PY{n+nt}{<dc:type}\PY{n+nt}{>}text\PY{n+nt}{</dc:type>}
			\PY{n+nt}{<dc:identifier}\PY{n+nt}{>}http://arxiv.org/abs/0704.0001\PY{n+nt}{</dc:identifier>}
			\PY{n+nt}{<dc:identifier}\PY{n+nt}{>}Phys.Rev.D76:013009,2007\PY{n+nt}{</dc:identifier>}
		\end{Verbatim}
	\end{block}
\end{frame}
\begin{frame}
	\frametitle{Parsen der Daten}
	\begin{itemize}
		\item  Parser in Python geschrieben 
		\item  kompletter Datensatz in den Speicher
			\begin{itemize}
				\item Overhead des XML-Parser nicht beachtet
			\end{itemize}
		\item iterativer Ansatz \footnote{http://www.ibm.com/developerworks/xml/library/x-hiperfparse/}
		\item benötigt ca. 70 Sekunden für 1.2 GB
	\end{itemize}
\end{frame}
\begin{frame}
	\frametitle{Verteilung der Themen}
	\begin{center}
		\includegraphics[scale=0.35]{../visual/treeParent.png}
	\end{center}
\end{frame}
\begin{frame}
	\frametitle{Aufschlüsselung von physics}
	\begin{center}
		\includegraphics[scale=0.45]{../visual/setSpecFreq.png}
	\end{center}
\end{frame}
\begin{frame}
	\frametitle{Häufigkeit von Themen pro Publikation}
	\begin{columns}
	        \column{.5\textwidth}
    \includegraphics[scale=0.35]{../visual/piechart.png}
		\column{.5\textwidth}
    \includegraphics[scale=0.25]{../visual/pieSubplot.png}
	\end{columns}
\end{frame}
\begin{frame}
	\frametitle{Was sind Assoziationsregeln?}
	\begin{itemize}
		\item Beschreibung
		\item Regel der Form "Wenn Menge A, dann Menge B"
		\item Kenngrößen
		\begin{itemize}
			\item Support - relative Häufigkeit der Regel in den Daten
	 		\item Konfidenz - 
	 		\item Lift
	\end{itemize}
	\end{itemize}
\end{frame}
\begin{frame}
	\frametitle{Assoziationsregeln - aller Themen}
	\begin{center}
	\begin{table}
	\rowcolors[]{1}{blue!20}{blue!10}
	\begin{tabular}{rccc}
		\tiny\textbf{Regel} &\tiny \textbf{Support} &\tiny \textbf{Konfidenz} & \tiny \textbf{Lift}\\
		\hline
		\tiny math $\implies$ stat & \tiny 0.6\% &\tiny 64\% &\tiny 3.0  \\
		\tiny physics:math-ph $\implies$ math &\tiny 3.8 \% &\tiny 100\% &\tiny 4.7 \\
		\tiny physics:hep-th, physics:math-ph  $\implies$ math &\tiny 0.9 \% &\tiny 100\% &\tiny 4.7 \\
		\tiny math, physics:hep-th  $\implies$ physics:math-ph  &\tiny 0.9 \% &\tiny 63\% &\tiny 16.3 \\
		\tiny physics:gr-qc, physics:hep-th $\implies$ physics:hep-th &\tiny 0.6 \% &\tiny 72 \% &\tiny 6.1 \\
		\tiny physics:gr-qc, physics:hep-th $\implies$ physics:astro-ph &\tiny 0.6 \% &\tiny 70 \% &\tiny 3.5 \\
		\tiny physics:gr-qc, physics:astro-ph $\implies$ physics:hep-th &\tiny 0.9 \% &\tiny 50 \%  &\tiny 4.3 \\
		\tiny physics:astro-ph, physics:hep-th $\implies$ physics:gr-qc  &\tiny 0.9 \% &\tiny 74 \% &\tiny 12.4 \\
	\end{tabular}
	 \caption*{Support: 0.5 \% und Konfidenz 50 \%}
	\end{table}
	\end{center}
\end{frame}
\begin{frame}
	\frametitle{Assoziationsregeln - Oberthemen}
	\begin{center}
	\begin{table}
	\rowcolors[]{1}{blue!20}{blue!10}
	\begin{tabular}{rccc}
		\tiny\textbf{Regel} &\tiny \textbf{Support} &\tiny \textbf{Konfidenz} & \tiny \textbf{Lift}\\
		\hline
		\tiny  $\emptyset \implies$ physics & \tiny 78\% &\tiny 78\% &\tiny 1.0  \\
		\tiny stat $\implies$ math  &\tiny 0.6 \% &\tiny 63 \% &\tiny 3.0 \\
		\tiny nlin $\implies$ physics  &\tiny 1.3 \% &\tiny 50 \% &\tiny 0.64 \\
		\tiny math, nlin $\implies$ physics  &\tiny 0.4 \% &\tiny 83 \% &\tiny 1.1 \\
	\end{tabular}
	 \caption*{Support: 0.1 \% und Konfidenz 50 \%}
	\end{table}
	\end{center}
\end{frame}
\begin{frame}
	\frametitle{Probleme}
	\begin{itemize}
		\item mehrere Datumsangaben
		\item Themenbereiche nicht uniform
		\item Aufwendig Themenbereiche nachzuschlagen
	\end{itemize}
\end{frame}
\begin{frame}
	\frametitle{Weitere Analysen}
	\begin{itemize}
		\item Aufschlüsselung der Themenbereiche
		\item Regeln für die Unterthemen
		\item Algorithmus implementieren?
			\begin{itemize}
				\item AIS-Algorithmnus
				\item Apriori-Algorithmus 
				\item FPGrowth 
			\end{itemize}
		\item Entwicklung in Abhängigkeit von der Zeit
	\end{itemize}
\end{frame}
\begin{frame}
	\frametitle{Beschreibung}
\begin{description}[Second Item]
	\item[First Item] Description of first item
	\item[Second Item] Description of second item
	\item[Third Item] Description of third item
	\item[Forth Item] Description of forth item
\end{description}
\end{frame}
\end{document}
