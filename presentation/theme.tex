%Copyright 2008 by Adrian Böhmichen
%
% This file is free software: you can redistribute it and/or modify
% it under the terms of the GNU General Public License as published by
% the Free Software Foundation, either version 3 of the License, or
% (at your option) any later version.
%
% This file is distributed in the hope that it will be useful,
% but WITHOUT ANY WARRANTY; without even the implied warranty of
% MERCHANTABILITY or FITNESS FOR A PARTICULAR PURPOSE.  See the
% GNU General Public License for more details.
%
% You should have received a copy of the GNU General Public License
% along with this file.  If not, see <http://www.gnu.org/licenses/>.

%%%%%%%%%%%%%%%%%%%%%%%%%%%%%%%%%%%%%%%%%%%%%%%%%%%%%%%%%%%%%%%%%
%     Ubuntuusers Vorlage für ein LaTeX-Beamer Theme            %
%                                                               %
% Für das Korrekte funktionieren benötigt man einen header.png  %
% und ein logo.png Datei!                                       %
% Zusätzlich muss man folgende Pakete benutzten:                %
%   \usepackage{graphicx}                                       %
%   \usepackage[percent]{overpic}                               %
%                                                               %
% Danach muss nur noch am Anfang die Datei                      %
% mit \input{} eingebunden werden.                              %
%                                                               %
%%%%%%%%%%%%%%%%%%%%%%%%%%%%%%%%%%%%%%%%%%%%%%%%%%%%%%%%%%%%%%%%%

%weitere Farbe spezifizieren:
%Farben von dem Humantheme
%\definecolor{Orange}{RGB}{240,165,19}
\definecolor{Orange}{RGB}{5,215,242}
%\definecolor{Human-Base}{RGB}{129,102,71}
\definecolor{Human-Base}{RGB}{5,25,242}
%Farben aus dem Inyokatheme
%\definecolor{uuheader1}{RGB}{164,143,101}
\definecolor{uuheader1}{RGB}{5,25,242}
%\definecolor{uuheader2}{RGB}{129,106,59}
\definecolor{uuheader2}{RGB}{5,25,242}


%Theme festlegen für alle Templates die nicht selbstständig definiert werden:
\usepackage{beamerthemedefault}


%Definieren des Innertheme, zuständig für die Symbole bei Listen
\setbeamertemplate{sections/subsections in toc}[square]
\setbeamertemplate{items}[circle]

\setbeamercolor{item}{fg=Human-Base}

%entfernen der Navigationsleiste
\beamertemplatenavigationsymbolsempty

%Logo definieren, man kann die Lage nicht verändern
%\logo{\includegraphics[scale=0.1]{logo.png}}


%Kopf- und Fußzeile definieren
%\setbeamertemplate{headline}
%{%
%\begin{overpic}[width=\paperwidth
% nächste Zeile dient zum anzeigen eines Rasters, für das paltzieren des ToC hilfreich
%,grid,tics=10
%]
%{header.png}%
%  \put(0,11){\insertsectionnavigationhorizontal{\paperwidth}{~}{~}}%
%  \end{overpic}
%}

\setbeamertemplate{footline}[text line]
{%
\begin{minipage}[b]{116mm}
\insertauthor \hfill%
%neue Navigationsleiste
 \insertframenumber ~/ \inserttotalframenumber\\[1ex]
\end{minipage}
}

% Farben festlegen ausserhalb des innertheme

%Allgemeine Angaben und Verbesserung vom default Theme
\setbeamercolor{structure}{fg=uuheader1}
\setbeamercolor{section in toc}{fg=Human-Base}
\setbeamercolor{subsection in toc}{parent=section in toc}
\setbeamercolor{framesubtitle}{fg=uuheader2}


%Farbe und Form der Blöcke definieren
\setbeamertemplate{blocks}[rounded]
%\setbeamercolor{block title}{fg=uuheader1,bg=Orange}
%\setbeamercolor{block title alerted}{use=alerted text,fg=black,bg=alerted text.fg!75!bg}
%\setbeamercolor{block title example}{use=example text,fg=black,bg=example text.fg!75!bg}

%\setbeamercolor{block body}{parent=normal text,use=block title,bg=block title.bg!25!bg}
%\setbeamercolor{block body alerted}{parent=normal text,use=block title alerted,bg=block title alerted.bg!25!bg}
%\setbeamercolor{block body example}{parent=normal text,use=block title example,bg=block title example.bg!25!bg}

%Für den Titleframe
\setbeamertemplate{title page}[default][rounded=true]
\setbeamercolor{title}{fg=uuheader2,bg=Orange}
